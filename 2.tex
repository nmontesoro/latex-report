\section{Contenidos de la entrega final}

\subsection{Modelo Entidad-Relación}

El esquema por el que nos decantamos es el de la fig. \ref{fig:der-2}. Los 
cambios principales son los siguientes:

\begin{itemize}

    \item Eliminamos la entidad \verb|DIRECCION|. Fue reemplazada por la 
    relación \verb|DIRIGE| con los atributos necesarios.

    \item Reemplazamos la agregación entre \verb|CONTRATADO| y \verb|TAREA|.
    La forma anterior tenía varias desventajas, entre ellas que no era posible
    determinar para qué área fue contratado cada empleado.

    \item Nos resultó más fácil de implementar las áreas sin distingur entre
    aquellas que son restringidas y las que no lo son. Ver más detalles en la
    sección \ref{sec:implementacion}.

    \item Más entidades tienen participaciones parciales en las relaciones. 
    Esto fue hecho principalmente para facilitar la carga de datos.

\end{itemize}

Las restricciones son similares a las de la primera entrega de este trabajo
práctico.

\begin{landscape}
    \begin{figure}[h]
        \centering
        \includegraphics[height=0.95\textheight]{/home/naza/Documents/drawing4.pdf}
        \caption{Diagrama Entidad-Relación final}
        \label{fig:der-2}
    \end{figure}
\end{landscape}

\subsection{Modelo lógico-relacional}

\textbf{Observación:} Ninguno de los atributos puede ser nulo, salvo que se 
indique lo contrario.

\subsubsection{Especialidad}

    \ent{especialidad}(\pk{\att{Id\_esp}}, \att{descripcion})

    \begin{itemize}
        \item \CK{\att{id\_esp}), (\att{descripcion}}.
        \item \PK{\att{id\_esp}}
        \item FK = $\emptyset$

        \item Normalización:
            
            \begin{itemize}
                \item \fmin{\att{id\_esp}}{\att{descripcion}}.
                \item \fmin{\att{descripcion}}{\att{id\_esp}}.
                    
                \item \att{id\_esp} es superclave y está a la izquierda, por lo
                    que está en BCNF.
                \item \att{descripcion} es superclave y está a la izquierda, por lo
                    que está en BCNF.
            \end{itemize}
    \end{itemize}

\subsubsection{Franja horaria}

    \ent{franja\_horaria}(\pk{\att{Id\_franja}}, \att{hora\_entrada}, \att{hora\_salida})

    \begin{itemize}
        \item \att{hora\_salida} debe ser mayor que \att{hora\_entrada}.
        \item \CK{\att{id\_franja}), (\att{hora\_entrada}, \att{hora\_salida}}.
        \item \PK{\att{id\_franja}}
        \item FK = $\emptyset$

        \item Normalización:
            
            \begin{itemize}
                \item \fmin{\att{id\_franja}}{\att{hora\_entrada}, \att{hora\_salida}}.
                \item \fmin{\att{hora\_entrada}, \att{hora\_salida}}{\att{id\_franja}}.
                    
                \item \att{id\_franja} es superclave y está a la izquierda, por lo
                    que está en BCNF.

                \item (\att{hora\_entrada}, \att{hora\_salida}) es superclave y
                    está a la izquierda, por lo
                    que está en BCNF.
            \end{itemize}
    \end{itemize}

\subsubsection{Seguridad}

    \ent{seguridad}(\pk{\att{id\_seg}}, \att{descripcion}, \att{nivel})

    \begin{itemize}
        \item \CK{\att{id\_seg}), (\att{descripcion}}.
        \item \PK{\att{id\_seg}}.
        \item FK = $\emptyset$.

        \item Normalización:
            \begin{itemize}
                \item \fmin{\att{id\_seg}}{\att{descripcion}, \att{nivel}}.
                \item \fmin{\att{descripcion}}{\att{id\_seg}, \att{nivel}}.
                \item \att{id\_seg} es superclave y está a la izquierda, por lo
                    que está en BCNF.
                \item \att{descripcion} es superclave y está a la izquierda, por lo
                    que está en BCNF.
            \end{itemize}
    \end{itemize}

\subsubsection{Tarea}

    \ent{tarea}(\pk{\att{id\_tarea}}, \att{descripcion})


    \begin{itemize}
        \item \CK{\att{id\_tarea}), (\att{descripcion}}.
        \item \PK{\att{id\_tarea}}
        \item FK = $\emptyset$

        \item Normalización:
            
            \begin{itemize}
                \item \fmin{\att{id\_tarea}}{\att{descripcion}}.
                \item \fmin{\att{descripcion}}{\att{id\_tarea}}.
                    
                \item \att{id\_tarea} es superclave y está a la izquierda, por lo
                    que está en BCNF.
                \item \att{descripcion} es superclave y está a la izquierda, por lo
                    que está en BCNF.
            \end{itemize}
    \end{itemize}


\subsubsection{Empleado}

    \ent{empleado}(\pk{\att{legajo}}, \att{nombre}, \att{apellido}, \att{f\_nac},
        \att{nro\_tel}, \att{huella}, \att{pwd}, \fk{\att{id\_seg}},
        \att{tipo\_empl}, \att{tipo\_prof}, \fk{\att{id\_esp}})

    \begin{itemize}
        \item \att{id\_seg} debe estar en \ent{seguridad}.\att{id\_seg}.
        \item \att{id\_esp} debe estar en \ent{especialidad}.\att{id\_esp}.
        \item \att{tipo\_empl} debe ser \verb|profesional|, \verb|no_profesional|
            o \verb|jerarquico|.
        \item \att{tipo\_prof} debe ser \verb|planta_permanente| o 
            \verb|contratado|.
        \item Si \att{tipo\_empl} es \verb|profesional| entonces 
            \att{tipo\_prof} e \att{id\_esp} no pueden ser nulos.
            En caso contrario, deben serlo.
        \item \CK{\att{legajo}), (\att{huella}}.
        \item \PK{\att{legajo}}.
        \item \FK{\att{id\_seg}), (\att{id\_esp}}.

        \item Normalización:
            
            \begin{itemize}

                \item \fmin{\att{legajo}}{\att{nombre}, \att{apellido}, 
                        \att{f\_nac}, \att{nro\_tel}, \att{huella},
                        \att{pwd}, \att{id\_seg}, \att{tipo\_empl},
                        \att{tipo\_prof}, \att{id\_esp}}.

                \item \fmin{\att{huella}}{\att{legajo}, \att{nombre}, \att{apellido}, 
                        \att{f\_nac}, \att{nro\_tel},
                        \att{pwd}, \att{id\_seg}, \att{tipo\_empl},
                        \att{tipo\_prof}, \att{id\_esp}}.
                    
                \item \att{legajo} es superclave y está a la izquierda, por lo
                    que está en BCNF.
                \item \att{huella} es superclave y está a la izquierda, por lo
                    que está en BCNF.
            \end{itemize}
    \end{itemize}


\subsubsection{Área}
    \ent{area}(\pk{\att{nro\_area}}, \att{nombre}, \att{F\_ini\_dir}, \fk{\att{id\_seg}}, 
        \fk{\att{Legajo\_dir}})

          \begin{itemize}
              \item \att{id\_seg} debe estar en \ent{seguridad}.\att{id\_seg}.
              \item \att{legajo\_dir} debe estar en \ent{empleado}.\att{legajo}
                  y el empleado al que hace referencia debe ser de tipo 
                  \verb|jerarquico|.
              \item \CK{\att{nro\_area}}.
              \item \PK{\att{nro\_area}}.
              \item \FK{\att{Id\_seg}), (\att{Legajo\_dir}}.
              \item Normalización:
                    \begin{itemize}
                        \item \fmin{\att{nro\_area}}{\att{nombre}, \att{F\_ini\_dir}, \att{id\_seg}, \att{Legajo\_dir}}
                        \item \att{nro\_area} es superclave y esta a la
                              izquierda, por lo que está en BCNF.
                    \end{itemize}
          \end{itemize}

\subsubsection{Evento}

    \ent{evento}(\pk{\att{nro\_evento}, \fk{\att{nro\_area}}}, \att{descripcion},
        \att{fecha})

    \begin{itemize}
        \item \att{nro\_area} debe estar en \ent{area}.\att{nro\_area}.
        \item \CK{\att{nro\_evento}, \att{nro\_area}}.
        \item \PK{\att{nro\_evento}, \att{nro\_area}}.
        \item \FK{\att{nro\_area}}.

        \item Normalización:

            \begin{itemize}
                \item \fmin{\att{nro\_evento}, \att{nro\_area}}{\att{descripcion}, \att{fecha}}

                \item (\att{nro\_evento}, \att{nro\_area}) es superclave y está
                    a la izquierda, por lo que está en BCNF.
            \end{itemize}
    \end{itemize}

\subsubsection{Entradas/salidas}

    \ent{entradas\_salidas}(\pk{\fk{\att{legajo}}, \fk{\att{nro\_area}}, \att{fecha}}, \att{es\_entrada}, \att{autorizado})

    \begin{itemize}
        \item Tanto \att{es\_entrada} como \att{autorizado} son \verb|true| o 
            \verb|false|.
        \item \att{fecha} incluye la hora.
        \item \att{legajo} debe estar en \ent{empleado}.\att{legajo}.
        \item \att{nro\_area} debe estar en \ent{area}.\att{nro\_area}.
        \item \CK{\att{legajo}, \att{nro\_area}, \att{fecha}}.
        \item \PK{\att{legajo}, \att{nro\_area}, \att{fecha}}.
        \item \FK{\att{legajo}), (\att{nro\_area}}.

        \item Normalización:

            \begin{itemize}
                \item \fmin{\att{legajo}, \att{nro\_area}, \att{fecha}}{\att{es\_entrada}, \att{autorizado}}.

                \item (\att{legajo}, \att{nro\_area}, \att{fecha}) es superclave 
                    y está a la izquierda, por lo que está en BCNF.
            \end{itemize}
    \end{itemize}

\subsubsection{Trabaja en}

    \ent{trabaja\_en}(\pk{\fk{\att{legajo}}, \fk{\att{nro\_area}}})

    \begin{itemize}
        \item \att{legajo} debe estar en \ent{empleado}.\att{legajo} y el 
            empleado al que hace referencia debe ser de tipo 
            \verb|no_profesional|. Además, el nivel de seguridad para el que se
            capacitó al empleado
            debe ser igual o mayor que el nivel de seguridad del área donde
            trabaja.
        \item \att{nro\_area} debe estar en \ent{area}.\att{nro\_area}.
        \item \CK{\att{legajo}, \att{nro\_area}}.
        \item \PK{\att{legajo}, \att{nro\_area}}.
        \item \FK{\att{legajo}), (\att{nro\_area}}.

        \item Normalización:

            \begin{itemize}
                \item BCNF (no tiene dependencias funcionales no triviales, por
                    lo que toda dependencia cumple que la parte izquierda de la
                    misma es superclave).
            \end{itemize}
    \end{itemize}

\subsubsection{Durante}

    \ent{durante}(\pk{\fk{\att{legajo}, \att{nro\_area}}, \fk{\att{id\_franja}}})

    \begin{itemize}
        \item (\att{legajo}, \att{nro\_area}) debe estar en \ent{trabaja\_en}.(\att{legajo}, \att{nro\_area}).
        \item \att{id\_franja} debe estar en \ent{franja\_horaria}.\att{id\_franja}.

        \item \CK{\att{legajo}, \att{nro\_area}, \att{id\_franja}}.
        \item \PK{\att{legajo}, \att{nro\_area}, \att{id\_franja}}.
        \item \FK{\att{legajo}, \att{nro\_area}), (\att{id\_franja}}.

        \item Normalización:

            \begin{itemize}
                \item BCNF (no tiene dependencias funcionales no triviales, por
                    lo que toda dependencia cumple que la parte izquierda de la
                    misma es superclave).
            \end{itemize}
    \end{itemize}

\subsubsection{Contrato}

    \ent{contrato}(\pk{\att{id\_contrato}}, \att{f\_ini}, \att{f\_fin},
        \fk{\att{id\_tarea}}, \fk{\att{nro\_area}}, \fk{\att{legajo}})

    \begin{itemize}
        \item \att{id\_tarea} debe estar en \ent{tarea}.\att{id\_tarea}.
        \item \att{nro\_area} debe estar en \ent{area}.\att{nro\_area}.
        \item \att{legajo} debe estar en \ent{empleado}.\att{legajo}. El 
            empleado al que hace referencia debe ser de tipo \verb|profesional|,
            y el tipo de profesional debe ser \verb|contratado|. Además, el 
            nivel de seguridad para el que se capacitó al empleado debe ser 
            mayor o igual que el nivel de seguridad del área para la cual se lo
            contrata.
        \item La fecha de finalización del contrato debe ser posterior a la 
            fecha de inicio (\att{f\_fin} > \att{f\_ini}).
        \item \CK{\att{id\_contrato}}
        \item \PK{\att{id\_contrato}}
        \item \FK{\att{id\_tarea}), (\att{nro\_area}), (\att{legajo}}

        \item Normalización:

            \begin{itemize}
                \item \fmin{\att{id\_contrato}}{\att{f\_ini}, \att{f\_fin}, \att{id\_tarea}, \att{nro\_area}, \att{legajo}}.

                \item (\att{id\_contrato}) es superclave y está a la izquierda, 
                    por lo que está en BCNF.
            \end{itemize}
    \end{itemize}


\subsubsection{Auditoría}
    \ent{auditoria}(\pk{\att{nro\_audit}}, \fk{\att{id\_contrato}}, 
          \att{descripcion}, \att{fecha}, \att{aprobado})

          \begin{itemize}
              \item \att{id\_contrato} debe estar en \ent{contrato}.\att{id\_contrato}.
              \item \ent{Auditoria}.\att{aprobada} es \{\texttt{true},
                    \texttt{false}\}.
              \item La fecha de la auditoría (\att{fecha}) debe estar dentro del
                  período del contrato al que hace referencia \att{id\_contrato}.
              \item \CK{\att{nro\_audit}, \att{id\_contrato}}.
              \item \PK{\att{nro\_audit}, \att{id\_contrato}}.
              \item \FK{\att{id\_contrato}}.
              \item Normalización:
                    \begin{itemize}
                        \item \fmin{\att{nro\_audit}, \att{id\_contrato}}{\att{descripcion}, \att{fecha}, \att{aprobado}}.
                        \item BCNF (\{\att{nro\_audit}, \att{id\_contrato}\} es superclave de
                              \ent{auditoria} y está del lado izquierdo de su dependencia funcional).
                    \end{itemize}
          \end{itemize}

\subsubsection{Estado de la base de datos}

Todas las entidades se encuentran en BCNF, entonces la base de datos entera se
encuentra en forma normal de Boyce-Codd.

\subsection{Implementación}
\label{sec:implementacion}

La implementación de la base de datos se hizo con MySQL. Se eligió este RDBMS 
por ser uno de los más populares y ampliamente utilizado en la industria, por su
facilidad de instalación y uso, por ser compatible con una amplia gama de 
lenguajes de programación y, además, por ser de código abierto y contar con una
versión gratuita.

La aplicación o \textit{frontend} se desarrolló con Blazor, un 
\textit{framework} de desarrollo web de Microsoft que permite crear aplicaciones
interactivas y de una sola página (SPA, \textit{single-page applications}) 
utilizando C\# en lugar de JavaScript.

Finalmente, se creó una imagen de Docker para poder crear la base de datos,
compilar la aplicación y ejecutarla con un sólo comando, y sin instalar 
dependencias adicionales. Todo el código se encuentra disponible en 
\url{https://github.com/nmontesoro/dbd}, junto con las instrucciones para su
ejecución y algunas capturas de pantalla de la aplicación web.

\subsubsection{Notas sobre la implementación}

Los niveles de seguridad se evalúan en base a \ent{seguridad}.\att{nivel}; el 
\att{id\_seg} es completamente independiente. Si bien esto agrega complejidad
a algunas consultas, simplifica el agregar nuevos niveles de seguridad (incluso
intermedios).

Los scripts de creación de la base de datos se encuentran en el directorio 
\texttt{./mysql-files}. Están comentados de manera de ayudar a la comprensión
del código y de definir a qué consigna responde cada \textit{trigger}, 
\textit{stored procedure} o \textit{constraint}.

El archivo \texttt{./Dockerfile.backend} es el encargado de crear la base de 
datos a partir de los scripts mencionados anteriormente, mientras que 
\texttt{./Dockerfile.frontend} compila la aplicación web y la ejecuta.
El archivo \texttt{./docker-compose.yml} es un archivo de configuración de 
Docker Compose que define dos servicios: uno para el \textit{backend} y otro 
para el \textit{frontend}. Cada servicio se construye a partir de su propio 
Dockerfile y se configura para exponer puertos específicos que puedan ser 
utilizados para acceder a los servicios desde el \textit{host}. Esto permite la
creación de un único contenedor con ambos servicios.

El contenedor de Docker está configurado para mostrar la aplicación web en el 
puerto 80, y para permitir el acceso directo a la base de datos mediante el 
puerto 3306, con usuario \verb|root| y contraseña \verb|PwdDbd|.

