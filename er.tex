\section{Resolución}
\subsection{Modelo entidad-relación}

El modelo entidad-relación que planteamos es el de la fig. \ref{fig:der}.
Asumimos las siguientes hipótesis:

\begin{itemize}
    \item La especialidad del Profesional es única. En caso de no serlo, se 
        constata la más pertinente al trabajo que realiza.
    \item Pueden existir Áreas que todavía no tengan ni empleados ni director 
        asignados.
    \item No todos los No Profesionales tienen instauradas franjas horarias 
        desde que son ingresados a la base de datos.
    \item Los No Profesionales ocupan una única franja horaria. En caso de no 
        existir, ésta se crea para asignársela.
    \item Los tipos de empleado son disjuntos. Un empleado no podrá ser de dos 
        tipos a la vez.
    \item El tipo de Profesional es disjunto. Un empleado de Planta Permanente 
        no podrá ser contratado y viceversa.
    \item El resultado de una auditoría es un booleano que puede o no venir 
        acompañado de una descripción breve.
\end{itemize}

\begin{landscape}
\begin{figure}[h]
    \centering
    \includegraphics[height=0.95\textheight]{/home/naza/Documents/drawing3.pdf}
    \caption{Diagrama Entidad-Relación}
    \label{fig:der}
\end{figure}
\end{landscape}

Las restricciones sobre el esquema de la fig. \ref{fig:der} son las siguientes:

\begin{itemize}
    \item Todo empleado debe ser jerarquico o profesional o no profesional. A su
        vez entre jerárquico, profesional y no profesional se debe obtener a 
        todos los empleados.
    \item Todo profesional debe ser contratado o de planta permanente. A su vez 
        entre planta permanente y contratado se debe obtener a todos los 
        empleados.
\end{itemize}

\section{Modelo lógico-relacional}

\newcommand{\pk}[1]{\underline{#1}}
\newcommand{\fk}[1]{\dashuline{#1}}
\newcommand{\ent}[1]{\uppercase{\texttt{#1}}}
\newcommand{\att}[1]{\texttt{\makefirstuc{#1}}}
\newcommand{\CK}[1]{CK=\{(#1)\}}
\newcommand{\FK}[1]{FK=\{(#1)\}}
\newcommand{\PK}[1]{PK=\{(#1)\}}
\newcommand{\fmin}[2]{$F_{\text{mín}}$: \{#1\} $\rightarrow$ \{#2\}}

\begin{itemize}
    \item \ent{Empleado}(\pk{\att{Legajo}}, \att{Nombre}, \att{Apellido}, 
        \att{F\_nac}, \att{Nro\_tel}, \att{Huella}, \att{Pwd}, \att{Tipo\_empl},
        \fk{\att{Id\_seg}})
        \begin{itemize}
            \item \ent{Empleado}.\att{Id\_seg} debe estar en 
                \ent{Seguridad}.\att{Id\_seg}.
            \item \ent{Empleado}.\att{Id\_seg} no puede ser nulo.
            \item \ent{Empleado}.\att{Tipo\_empl} es \{0 (\ent{profesional}), 1 
                    (\ent{no\_profesional}), 2 (\ent{jerarquico})\}.
            \item \ent{Empleado}.\att{Tipo\_empl} no puede ser nulo.
            \item \CK{\att{legajo}}.
            \item \PK{\att{legajo}}.
            \item \FK{\att{Id\_seg}}.
            \item Normalización:
                \begin{itemize}
                    \item \fmin{\att{legajo}}{\att{Nombre}, \att{Apellido}, 
                            \att{F\_nac}, \att{Nro\_tel}, \att{Huella}, 
                            \att{Pwd}, \att{Tipo\_empl}, \att{Id\_seg}}.
                    \item BCNF({\att{legajo}} es superclave de \ent{empleado}, 
                        está del lado izquierdo de su dependencia funcional).
                \end{itemize}
        \end{itemize}

    \item \ent{seguridad}(\pk{\att{Id\_seg}}, \att{Desc}, \att{Nivel})
        \begin{itemize}
            \item \ent{seguridad}.\att{id\_seg} debe estar en 
                \ent{area\_restringida}.\att{id\_seg}.
            \item Los niveles de seguridad se ordenan según \att{nivel}. Con 
                esto se permite añadir nuevos niveles, incluso intermedios, sin
                modificar el resto de las instancias.
            \item \CK{\att{id\_seg}}.
            \item \PK{\att{id\_seg}}.
            \item \FK{$\emptyset$}.
            \item Normalización:
                \begin{itemize}
                    \item \fmin{\att{id\_seg}}{\att{desc}, \att{nivel}}.
                    \item BCNF(\{\att{id\_seg}\} es superclave de 
                        \ent{seguridad} y está del lado izquierdo de su 
                        dependencia funcional.
                \end{itemize}
        \end{itemize}
        
    \item \ent{profesional}(\pk{\fk{\att{legajo}}}, \att{tipo\_prof}, 
        \fk{\att{id\_esp}})
            \begin{itemize}
                \item \ent{profesional}.\att{tipo\_prof} es \{0 
                        (\ent{planta\_permanente}), 1 (\ent{contratado})\}.
                \item \ent{profesional}.\att{legajo} debe estar en 
                    \ent{empleado}.\att{legajo}.
                \item \ent{profesional}.\att{id\_esp} debe estar en 
                    \ent{especialidad}.\att{id\_esp}.
                \item Para \ent{profesional}.\att{legajo} == 
                    \ent{empleado}.\att{legajo}, 
                    \ent{empleado}.\att{tipo\_emp} = 0.
                \item \CK{\att{legajo}}.
                \item \PK{\att{legajo}}.
                \item \FK{\att{id\_seg}}.
                \item Normalización:
                    \begin{itemize}
                        \item \fmin{\att{legajo}}{\att{tipo\_prof}, 
                                \att{id\_esp}}.
                        \item BCNF(\{\att{legajo}\} es superclave de 
                            \ent{profesional} y está del lado izquierdo de su 
                            dependencia funcional.
                    \end{itemize}
            \end{itemize}
            
        \item \ent{especialidad}(\pk{\att{id\_esp}}, \att{desc})
            \begin{itemize}
                \item \ent{especialidad}.\att{id\_esp} debe estar en 
                    \ent{profesional}.\att{id\_esp}.
                \item \CK{\att{id\_esp}}.
                \item \PK{\att{id\_esp}}.
                \item \FK{$\emptyset$}.
                \item Normalización:
                    \begin{itemize}
                        \item \fmin{\att{id\_esp}}{\att{desc}}.
                        \item BCNF(\{\att{legajo}\} es superclave de 
                            \ent{especialidad} y está del lado izquierdo de su 
                            dependencia funcional.
                    \end{itemize}
            \end{itemize}

        \item \ent{planta\_permanente}(\pk{\fk{\att{legajo}}}, 
            \fk{\att{nro\_area}})
            \begin{itemize}
                \item \ent{planta\_permanente}.\att{legajo} debe estar en 
                    \ent{profesional}.\att{legajo}.
                \item \ent{planta\_permanente}.\att{nro\_area} debe estar en
                    \ent{area}.\att{nro\_area}.
                \item Para \ent{planta\_permanente}.\att{legajo} == 
                    \ent{profesional}.\att{legajo}, 
                    \ent{profesional}.\att{tipo\_prof} = 0.
                \item \ent{empleado}.\att{id\_seg} de 
                    \ent{planta\_permanente}.\att{legajo} $\geq$ 
                    \ent{area\_restringida}.\att{seguridad} de 
                    \ent{planta\_permanente}.\att{nro\_area}.
                \item \CK{\att{legajo}}.
                \item \PK{\att{legajo}}.
                \item \FK{\att{nro\_area}}.
                \item Normalización:
                    \begin{itemize}
                        \item \fmin{\att{legajo}}{\att{nro\_area}}.
                        \item BCNF(\{\att{legajo}\} es superclave de 
                            \ent{empleado} y está del lado izquierdo de su
                            dependencia funcional).
                    \end{itemize}
            \end{itemize}

        \item \ent{contratado}(\pk{\fk{\att{legajo}}})
            \begin{itemize}
                \item \ent{contratado}.\att{legajo} debe estar en 
                    \ent{profesional}.\att{legajo}.
                \item Para \ent{contratado}.\att{legajo} == 
                    \ent{profesional}.\att{legajo},
                    \ent{profesional}.\att{tipo\_prof} = 1.
                \item \CK{\att{legajo}}.
                \item \PK{\att{legajo}}.
                \item \FK{$\emptyset$}.
                \item Normalización:
                    \begin{itemize}
                        \item BCNF(no tiene dependencias funcionales no 
                            triviales, por lo que toda dependencia cumple que
                            la parte izquierda de la misma es superclave).
                    \end{itemize}
            \end{itemize}

        \item \ent{desarrolla}(\pk{\fk{\att{legajo}}}, 
            \pk{\fk{\att{id\_tarea}}}, \att{f\_ing}, \att{f\_fin})
            \begin{itemize}
                \item \ent{desarrolla}.\att{legajo} debe estar en 
                    \ent{contratado}.\att{legajo}.
                \item \ent{desarrolla}.\att{id\_tarea} debe estar en 
                    \ent{tarea}.\att{id\_tarea}.
                \item \ent{desarrolla}.\att{f\_fin} $>$ 
                    \ent{desarrolla}.\att{f\_ing}.
                \item \CK{\att{legajo}, \att{id\_tarea}}
                \item \PK{\att{legajo}, \att{id\_tarea}}
                \item \FK{\att{legajo}), (\att{id\_tarea}}
                \item Normalización:
                    \begin{itemize}
                        \item \fmin{\att{legajo}, \att{id\_tarea}}{\att{f\_ing},
                                \att{f\_fin}}.
                        \item BCNF(\{\att{legajo}, \att{id\_tarea}\} es 
                            superclave de \ent{desarrolla} y está del lado 
                            izquierdo de su dependencia funcional).
                    \end{itemize}
            \end{itemize}

        \item \ent{auditoria}(\fk{\att{legajo}}, \fk{\att{id\_tarea}}, 
            \fk{\att{nro\_audit}}, \att{desc}, \att{fecha}, \att{aprobada})
            \begin{itemize}
                \item \ent{Auditoria}.\att{id\_tarea} debe estar 
                    \ent{Desarrolla}.\att{id\_tarea}.
                \item \ent{Auditoria}.\att{legajo} debe estar en 
                    \ent{Desarrolla}.\att{legajo}.
                \item \ent{Desarrolla}.\att{(legajo, id\_tarea)} puede estar (o 
                    no) en \ent{Auditoria}.\att{(legajo, id\_tarea)}
                \item \ent{Auditoria}.\att{aprobada} es \{\texttt{true}, 
                    \texttt{false}\}.
                \item \CK{\att{legajo}, \att{id\_tarea}, \att{nro\_audit}}.
                \item \PK{\att{legajo}, \att{id\_tarea}, \att{nro\_audit}}.
                \item \FK{\att{legajo}, \att{id\_tarea}}.
                \item Normalización:
                    \begin{itemize}
                        \item \fmin{\att{legajo}, \att{id\_tarea}, 
                                \att{nro\_audit}}{\att{desc}, \att{fecha}, 
                                \att{aprobada}}.
                        \item BCNF(\{\att{legajo}, \att{id\_tarea}, 
                        \att{nro\_audit}\} es superclave de \ent{auditoria} y 
                            está del lado izquierdo de su dependencia 
                            funcional).
                    \end{itemize}
            \end{itemize}

        \item \ent{tarea}(\pk{\att{id\_tarea}}, \att{desc}, 
            \fk{\att{nro\_area}})
            \begin{itemize}
                \item \ent{tarea}.\att{nro\_area} debe estar en 
                    \ent{area}.\att{nro\_area}.
                \item \ent{area}.\att{nro\_area} puede estar (o no) en 
                    \ent{tarea}.\att{nro\_area} (esto va en \ent{area}).
                \item \CK{\att{id\_tarea}}.
                \item \PK{\att{id\_tarea}}.
                \item \FK{\att{nro\_tarea}}.
                \item Normalización:
                    \begin{itemize}
                        \item \fmin{\att{id\_tarea}}{\att{desc}, \att{nro\_area}}.
                        \item BCNF(\{\att{id\_tarea}\} es superclave de 
                            \ent{tarea} y está del lado izquierdo de su 
                            dependencia funcional.
                    \end{itemize}
            \end{itemize}


        \item \ent{no\_profesional}(\pk{\fk{\att{legajo}}})
            \begin{itemize}
                \item \ent{no\_profesional}.\att{legajo} debe estar en 
                    \ent{empleado}.\att{legajo}.
                \item Para \ent{no\_profesional}.\att{legajo} == 
                    \ent{empleado}.\att{legajo}, 
                    \ent{empleado}.\att{tipo\_empl} = 1.
                \item \ent{no\_profesional}.\att{legajo} debe estar en 
                    \ent{trabaja\_en}.\att{legajo}.
                \item \CK{\att{legajo}}.
                \item \PK{\att{legajo}}.
                \item \PK{$\emptyset$}.
                \item Normalización:
                    \begin{itemize}
                        \item BCNF(no tiene dependencias funcionales no 
                            triviales, por lo que toda dependencia cumple que
                            la parte izquierda de la misma es superclave).
                    \end{itemize}
            \end{itemize}

        \item \ent{trabaja\_en}(\pk{\fk{\att{legajo}}}, \pk{\fk{\att{nro\_area}}},
            \fk{\att{id\_franja}})
            \begin{itemize}
                \item \ent{Trabaja\_en}.(\att{legajo}, \att{nro\_area}) debe 
                    estar en \ent{No\_profesional}.\att{legajo}
	            \item \ent{Trabaja\_en}.\att{nro\_area} debe estar en 
                    \ent{Area}.\att{nro\_area}.
                \item \ent{Trabaja\_en}.\att{id\_franja} debe estar en 
                    \ent{Franja\_horaria}.\att{id\_franja} o ser nulo.
                \item \ent{Empleado}.\att{id\_seg} de 
                    \ent{trabaja\_en}.\att{legajo} $\geq$ 
                    \ent{area}.\att{seguridad} de 
                    \ent{trabaja\_en}.\att{nro\_area}
                \item \ent{Trabaja\_en}.\att{legajo} == 
                    \ent{Trabaja\_en}.\att{legajo} con \ent{Trabaja\_en} <> 
                    \ent{Trabaja\_en}, entonces \ent{Trabaja\_en}.\att{franja\_horaria} <> \ent{Trabaja\_en}.\att{franja\_horaria}.
                \item \CK{\att{legajo}, \att{nro\_area}, \att{id\_franja}}.
                \item \PK{\att{legajo}, \att{nro\_area}, \att{id\_franja}}.
                \item \FK{\att{legajo}), (\att{nro\_area}), (\att{id\_franja}}.
                \item Normalización:
                    \begin{itemize}
                        \item \fmin{\att{legajo}, \att{id\_tarea}}{\att{id\_franja}}.
                        \item \fmin{\att{legajo}, \att{id\_franja}}{\att{nro\_area}}.
                        \item BCNF(\{\att{legajo}, \att{nro\_area}\} y 
                            \{\att{legajo}, \att{id\_franja}\} son superclave de 
                            \ent{trabaja\_en} y ambas están del lado izquierdo 
                            de su dependencia funcional)
                    \end{itemize}
            \end{itemize}


        \item \ent{franja\_horaria}(\pk{\att{id\_franja}}, \att{hora\_entrada},
            \att{hora\_salida})
            \begin{itemize}
                \item \ent{franja\_horaria}.\att{id\_franja} debe estar en 
                    \ent{trabaja\_en}.\att{id\_franja}.
                \item \ent{franja\_horaria}.\att{hora\_entrada} < 
                    \ent{franja\_horaria}.\att{hora\_salida}.
                \item \CK{\att{id\_franja}\}, (\att{hora\_entrada}, 
                        \att{hora\_salida}}.
                \item \PK{\att{id\_franja}}
                \item \FK{$\emptyset$}
                \item Normalización:
                    \begin{itemize}
                        \item \fmin{\att{hora\_entrada}, 
                                \att{hora\_salida}}{\att{id\_franja}}.
                        \item \fmin{\att{id\_franja}}{\att{hora\_entrada}, 
                                \att{hora\_salida}}.
                        \item BCNF(\{\att{hora\_entrada}, \att{hora\_salida}\}
                            e \att{id\_franja} son superclave de 
                            \ent{franja\_horaria} y ambas están del lado 
                            izquierdo de su dependencia funcional.
                    \end{itemize}
            \end{itemize}

        \item \ent{jerarquico}(\pk{\fk{\att{legajo}}})
            \begin{itemize}
                \item \ent{jerarquico}.\att{legajo} debe estar en
                    \ent{empleado}.\att{legajo}.
                \item Para \ent{jerarquico}.\att{legajo} == 
                    \ent{empleado}.\att{legajo}, \ent{empleado}.\att{tipo\_empl}
                    = 2.
                \item \ent{jerarquico}.\att{legajo} puede no estar en 
                    \ent{ocupa}.\att{legajo}.
                \item \CK{\att{legajo}}.
                \item \PK{\att{legajo}}.
                \item \FK{$\emptyset$}.
                \item Normalización:
                    \begin{itemize}
                        \item BCNF(no tiene dependencias funcionales no 
                            triviales, por lo que toda dependencia cumple que la
                            parte izquierda de la misma es superclave)
                    \end{itemize}
            \end{itemize}

        \item \ent{direccion}(\pk{\att{id\_dir}}, \att{f\_ini}, \att{f\_fin},
                              \fk{\att{nro\_area}})
            \begin{itemize}
                \item \ent{direccion}.\att{id\_dir} puede no estar en 
                    \ent{ocupa}.\att{id\_dir}.
                \item \ent{direccion}.\att{nro\_area} debe estar en 
                    \ent{area}.\att{nro\_area}.
                \item \CK{\att{id\_dir}}.
                \item \PK{\att{id\_dir}}.
                \item \FK{\att{nro\_area}}.
            \end{itemize}

        \item \ent{ocupa}(\pk{\fk{\att{legajo}}}, \pk{\fk{\att{id\_dir}}})
            \begin{itemize}
                \item \ent{ocupa}.\att{legajo} debe estar en 
                    \ent{jerarquico}.\att{legajo}.
                \item \ent{ocupa}.\att{id\_dir} debe estar en 
                    \ent{direccion}.\att{id\_dir}.
                \item \CK{(\att{legajo}, \att{id\_dir})}.
                \item \PK{(\att{legajo}, \att{id\_dir})}.
                \item \FK{\att{legajo}, \att{id\_dir}}.
            \end{itemize}

        \item \ent{area}(\pk{\att{nro\_area}}, \att{nombre}, 
            \att{es\_restringida})
            \begin{itemize}
                \item \ent{area}.\att{nro\_area} puede estar (o no) en 
                    \ent{direccion}.\att{nro\_area}.
                \item \ent{area}.\att{nro\_area} puede estar (o no) en 
                    \ent{tarea}.\att{nro\_area}.
                \item \ent{area}.\att{nro\_area} puede estar (o no) en 
                    \ent{trabaja\_en}.\att{nro\_area}.
                \item \ent{area}.\att{nro\_area} puede estar (o no) en 
                    \ent{planta\_permanente}.\att{nro\_area}.
                \item \ent{area}.\att{es\_restringida} es \{\texttt{true}, 
                        \texttt{false}\}.
                \item \CK{\att{nro\_area}}.
                \item \PK{\att{nro\_area}}.
                \item \FK{$\emptyset$}.
            \end{itemize}


        \item \ent{area\_restringida}(\pk{\fk{\att{nro\_area}}}, \fk{\att{id\_seg}})
            \begin{itemize}
                \item \ent{area\_restringida}.\att{nro\_area} debe estar en 
                    \ent{area}.\att{nro\_area}.
                \item \ent{area\_restringida}.\att{id\_seg} debe estar en 
                    \ent{seguridad}.\att{id\_seg}.
                \item Para \ent{empleado}.\att{legajo} == 
                    \ent{area}.\att{legajo} y \ent{area}.\att{nro\_area} en 
                    \ent{area\_restringida}.\att{nro\_area},
                    \ent{empleado}.\att{id\_seg} $\geq$ 
                    \ent{area\_restringida}.\att{id\_seg}.
                \item \CK{\att{nro\_area}}.
                \item \PK{\att{nro\_area}}.
                \item \FK{\att{nro\_area}, \att{id\_seg}}.
                \item Normalización:
                    \begin{itemize}
                        \item \fmin{\att{nro\_area}}{\att{seguridad}}.
                        \item BCNF(\{\att{nro\_area}\} es superclave de 
                            \ent{area\_restringida} y está del lado izquierdo
                            de su dependencia funcional.
                    \end{itemize}
            \end{itemize}


        \item \ent{evento}(\pk{\fk{\att{nro\_area}}}, \pk{\att{nro\_evento}}, 
            \att{desc}, \att{fecha})
            \begin{itemize}
                \item \ent{evento}.\att{nro\_area} debe estar en 
                    \ent{area}.\att{nro\_area}.
                \item \ent{area}.\att{nro\_area} puede estar (o no) en 
                    \ent{evento}.\att{nro\_area}.
                \item \CK{\att{nro\_evento}}.
                \item \PK{\att{nro\_evento}}.
                \item \FK{\att{nro\_area}}.
                \item Normalización:
                    \begin{itemize}
                        \item \fmin{\att{nro\_area}, \att{nro\_evento}}
                            {\att{desc}, \att{fecha}}.
                        \item BCNF(\{\att{nro\_area}, \att{nro\_evento}\} es
                            superclave de \ent{evento} y está del lado izquierdo
                            de su dependencia funcional.
                    \end{itemize}
            \end{itemize}


        \item \ent{entrada\_salida}(\pk{\fk{\att{legajo}}}, 
            \pk{\fk{\att{nro\_area}}}, \att{fecha}, \att{entrada}, 
            \att{autorizado})
            \begin{itemize}
                \item \ent{entrada\_salida}.\att{legajo} debe estar en
                    \ent{empleado}.\att{legajo}.
                \item \ent{entrada\_salida}.\att{nro\_area} debe estar en 
                    \ent{area\_restringida}.\att{nro\_area}.
                \item Si \ent{entrada\_salida}.\att{legajo} está en 
                    \ent{planta\_permanente}.\att{legajo} entonces 
                    \att{legajo} debe determinar funcionalmente a \att{nro\_area}.
                \item \CK{\att{legajo}, \att{nro\_area}}.
                \item \PK{\att{legajo}, \att{nro\_area}}.
                \item \FK{\att{legajo}), (\att{nro\_area}}.
            \end{itemize}

        \item \ent{durante}(\pk{\fk{\att{legajo}}}, \pk{\fk{\att{nro\_area}}}, \pk{\fk{\att{id\_franja}}})
            \begin{itemize}
                \item (\ent{durante}.\att{legajo}, 
                    \ent{durante}.\att{nro\_area}) puede no estar en 
                    (\ent{trabaja\_en}.\att{legajo}, 
                    \ent{trabaja\_en}.\att{nro\_area}).
                \item \ent{durante}.\att{id\_franja} debe estar en 
                    \ent{franja\_horaria}.\att{id\_franja}.
                \item \CK{\att{legajo}, \att{nro\_area}, \att{id\_franja}}.
                \item \PK{\att{legajo}, \att{nro\_area}, \att{id\_franja}}.
                \item \FK{\att{legajo}, \att{nro\_area}, \att{id\_franja}}.
            \end{itemize}
\end{itemize}
