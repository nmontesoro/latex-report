\documentclass[spanish, 12pt]{article}
\usepackage[T1]{fontenc} % https://tex.stackexchange.com/questions/664/why-should-i-use-usepackaget1fontenc
\usepackage[spanish,es-tabla]{babel} % Traducciones, y cambia cuadro por tabla
\usepackage{csquotes} % Mejor soporte de comillas en bibliografía
\usepackage{circuitikz} % Para armar circuitos directamente en LaTeX
\usepackage{amsmath} % Mejoras en fórmulas matemáticas
\usepackage{multirow} % Multples filas 
\usepackage{multicol} % Múltiples columnas
\usepackage[a4paper, portrait, margin=2.5cm]{geometry} % Ajusto márgenes y tamaño de página
\usepackage{booktabs} % Mejoras en tablas
\usepackage{graphicx} % Extended interface for graphics inclusion
\usepackage{float} % Mejor soporte de float para figuras
\usepackage{cancel} % Permite cancelar términos

% --- Para listar código
\usepackage{listings} 
\usepackage{xcolor}
% --- Para figuras y subfiguras
\usepackage{caption}
\usepackage{subcaption}
% --- Para el título
\usepackage[absolute,overlay]{textpos}
% --- Para hipervínculos
\usepackage{hyperref}
\hypersetup{
    hidelinks = true
}

\urlstyle{same}

% --- Colores para listados de código
\definecolor{codegreen}{rgb}{0,0.6,0}
\definecolor{codegray}{rgb}{0.5,0.5,0.5}
\definecolor{codepurple}{rgb}{0.58,0,0.82}
\definecolor{backcolour}{rgb}{0.95,0.95,0.92}

\lstdefinestyle{mystyle}{
    commentstyle=\color{codegreen},
    keywordstyle=\color{magenta},
    numberstyle=\tiny\color{codegray},
    stringstyle=\color{codepurple},
    basicstyle=\ttfamily\footnotesize,
    breakatwhitespace=false,
    breaklines=true,
    captionpos=b,
    keepspaces=true,
    numbers=left,
    numbersep=5pt,
    showspaces=false,
    showstringspaces=false,
    showtabs=false,
    tabsize=2
}

\lstset{style=mystyle}
% ---

% --- Configuración de bibiliografía
\usepackage[
backend=biber,
style=alphabetic,
sorting=ynt
]{biblatex}
\addbibresource{references.bib}
% ---

% --- Para informes de tipo pregunta/respuesta
\newcounter{question}
\setcounter{question}{0}
\newcommand{\question}[1]{
    \stepcounter{question}
    \vspace{0.2cm}
    \noindent\rule{\textwidth}{0.4mm}
    {\large\thequestion. #1\par}
    \noindent\rule{\textwidth}{0.4mm}
}
\newcommand{\answer}[1]{
    \vspace{0.2cm}
    #1
}

% --- Unidades
\usepackage{siunitx}
\sisetup{
    separate-uncertainty=true,
    round-mode = uncertainty,
    round-precision = 1,
    locale = DE
}

% Graficos con gnuplot
\usepackage{tikzscale}
\usepackage{gnuplot-lua-tikz}

\usepackage{wrapfig}

% --- Para TODOs y comentarios
\newcommand{\TODO}[1]{
    \begin{center}
        \colorbox{orange}{{\Huge TODO: }#1}
    \end{center}
}

\newcommand{\comment}[1]{
    \begin{center}
        \colorbox{yellow}{{\Huge TODO: }#1}
    \end{center}
}

% Graficos con gnuplot
\usepackage{tikzscale}
\usepackage{gnuplot-lua-tikz}

\usepackage{wrapfig}

\begin{document}

    \newcommand{\materia}[1]{{\huge #1}\\\vspace{0.5cm}}
\newcommand{\titulo}[1]{{\Huge \textbf{#1}}\\\vspace{0.5cm}}
\newcommand{\subtitulo}[1]{{\huge #1}\\\vspace{0.5cm}}
\newcommand{\autor}[3]{#1\\{\small Mat. #2\\\href{mailto:#3}{#3}}\\\vspace{0.7cm}}
\newcommand{\fecha}[4]{{\small Fecha de #1: #4-#3-#2}\\} % YYYY-MM-DD
\newcommand{\departamento}[1]{{\small Departamento de #1\\Facultad de Ingeniería\\Universidad Nacional de Mar del Plata}}
\newenvironment{mytitlepage}
    {\pagenumbering{gobble}\begin{center}}
    {\end{center}\newpage\pagenumbering{arabic}}

\begin{mytitlepage}
    \includegraphics[width=4cm]{img/logo-fi.pdf}
   
    \begin{textblock*}{\paperwidth}(0cm, 7cm)
    \materia{Diseño de Bases de Datos}
    \titulo{Trabajo Práctico}
    %\subtitulo{Trabajo de Laboratorio I}
    \end{textblock*}
    
    \begin{textblock*}{\paperwidth}(0cm, 12cm)
    \autor{Nazareno Montesoro}{13.668}{n.montesoro@gmail.com}
    \autor{Santiago Castiñeira}{15.907}{santiescasti@gmail.com}
    \end{textblock*}
    
    \begin{textblock*}{\paperwidth}(0cm, 25cm) 
    %\fecha{realización}{06}{09}{2022}
    \fecha{primera entrega}{26}{04}{2023}
    \fecha{entrega final}{21}{06}{2023}
    \end{textblock*}
    
    \begin{textblock*}{\paperwidth}(0cm, 27cm) 
    \departamento{Electrónica y Computación}
    \end{textblock*}
\end{mytitlepage}

    \tableofcontents

    \newpage
    \section{Resumen}

En este trabajo de laboratorio se determinará la constante de la gravedad $g$ a
partir de la oscilación de un péndulo físico de longitud variable. Se analizará
también la influencia de los errores aleatorios en los cálculos. 

    \section{Objetivos}

\subsection{Objetivos generales}

\begin{itemize}
    \item Mejorar nuestra comprensión acerca del funcionamiento de los
    amplificadores operacionales y de los circuitos integrados.
    \item Interpretar el modo de montaje de los circuitos integrados mediante
    su hoja de datos o \textit{datasheet}.
    \item Reforzar nuestras habilidades tanto en el montaje de circuitos en 
    \textit{protoboard} como en el uso de instrumental de laboratorio.
\end{itemize}

\subsection{Objetivos particulares}

\begin{itemize}
    \item Verificar empíricamente las ecuaciones de ganancia de tensión para 
    los amplificadores inversores y no inversores
    \item Verificar el funcionamiento del amplificador operacional configurado
    como seguidor de tensión
    \item Verificar las características del amplificador operacional que 
    permiten considerarlo como un elemento ideal al analizar circuitos.
    \item Detectar alguna diferencia entre el funcionamiento del amplificador 
    operacional físico y su modelo ideal correspondiente.
\end{itemize}


    \section{Introducción y fundamentos}

En este práctico Ud. determinará el valor de la constante de la gravedad
utilizando un péndulo físico de varilla rígida. El período de la oscilación $T$
está relacionado con la longitud $L$ de la varilla y la constante de la
gravedad $g$ por medio de la siguiente ecuación:

\begin{equation}
    \label{ec:intro:periodo}
    T = 2\pi \sqrt{ \frac{2L}{3g} }
\end{equation}

La medida del período $T$ estará afectada por errores aleatorios, que luego 
de su análisis por el método estadístico se propagarán al valor de la
constante $g$.


    \newpage
    \section{Métodos y procedimiento}

Los elementos a utilizar son:

\begin{itemize}
    \item Una placa de experimentación o \textit{protoboard}
    \item Un circuito integrado TL072
    \item Resistores de \SI{1}{\kilo\ohm}, \SI{10}{\kilo\ohm},
        \SI{15}{\kilo\ohm} y \SI{47}{\kilo\ohm} (tres de cada valor)
    \item Una batería de \SI{9}{\volt} con clip para conectarla
    \item Multímetro digital
    \item Osciloscopio digital
    \item Generador de funciones
    \item Fuente de alimentación de laboratorio
    \item Alicate
    \item Cables para \textit{protoboard}
\end{itemize}

El procedimiento para ambos casos consiste en montar los circuitos según su 
correspondiente esquemático en una \textit{protoboard} para medir las tensiones
de entrada y de salida del operacional, las tensiones en las terminales inversora y
no inversora (y las diferencias de potencial entre éstas) y las corrientes en
cada resistencia, haciendo uso del multímetro configurado en las escalas
correspondientes.


    \newpage
    \section{Primera parte: amplificador inversor}

Se arma el circuito de la fig. \ref{fig:1:esquema} en un \textit{protoboard}
utilizando resistencias $R_1 = \SI{10}{\kilo\ohm}$, $R_2 = \SI{15}{\kilo\ohm}$ y
$R_3 = \SI{10}{\kilo\ohm}$. Luego se miden las tensiones $V_i$, $V_o$, $V_{+}$,
$V_{-}$ y $V_{+} - V_{-}$, y las corrientes que atraviesan a las resistencias
para valores de $R_L$ de 1, 10 y \SI{47}{\kilo\ohm}. 

\subsection{Resolución teórica}

Considerando al operacional como un elemento ideal (ver sección \ref{sec:intro})
se sabe que su resistencia de entrada es infinita. Por tal motivo no circulará
corriente a través de la resistencia $R_3$, lo cual implica que no habrá caída
de tensión alguna entre sus terminales. Efectivamente es como tener el circuito
descrito en la sección \ref{sec:intro:opamp-inversor} y por ende se puede usar
la ecuación \ref{ec:intro:opamp-inversor}. Entonces queda:

\begin{equation}
    \label{ec:1-teoria:ganancia}
    A = -\frac{R_2}{R_1}
\end{equation}

\begin{equation}
    \label{ec:1-teoria:err-ganancia}
    \Delta A = \left| - \frac{1}{R_1} \right| \Delta R_2
             + \left| - \frac{R_2}{{R_1}^2} \right| \Delta R_1
\end{equation}

Se observa que el valor de la ganancia \textbf{no depende del valor de la
resistencia de carga}. Considerando que $\Delta R_i = 0.2\,R_i\ \forall\ i$, ya
que los resistores utilizados tienen una tolerancia de $\pm 20\%$, el análisis
anterior resulta en:

\[
    \input{src/1/snippets/ganancia-analitica}
\]

Como puede observarse, el valor de la ganancia no debería verse afectado por el
resistor $R_3$. Esto se debe a la alta resistencia de entrada del operacional
(que de hecho es infinita en este modelo ideal).

Si bien el modelo ideal no admite que circule corriente entre masa y la
terminal no inversora, debido a esta resistencia infinita, en la realidad
el operacional tiene una ganancia \textit{finita}, pero muy grande (en torno a
los $10^{12}$ \si{\ohm} en el caso del TL072 \cite[pág. 5]{datasheet-tl072}).
Por este motivo se produce una corriente mínima, llamada corriente de
polarización (\textit{input bias current}), que depende del dispositivo que se
utilice y que provoca una caída de potencial entre los terminales de $R_3$.
Para el TL072, esta corriente está entre 65 y \SI{200}{\pico\ampere} 
\cite[pág. 5]{datasheet-tl072}. Es por esta razón que se aconseja no
conectar la terminal no inversora directamente a masa \cite[pág. 252]{AOE}.


\subsection{Datos obtenidos}

Se midieron los siguientes valores de resistencias:

\begin{itemize}
        \input{src/1/snippets/resistencias}
\end{itemize}

Los valores medidos de $R_L$, comparados con sus valores nominales, se encuentran en la tabla \ref{tab:1-datos:resistencias-l}.

También se tomaron mediciones de las distintas tensiones presentes en el circuito (tablas \ref{tab:1-datos:tensiones-1} y \ref{tab:1-datos:tensiones-2}) que fueron utilizadas para calcular las corrientes que circulan por cada resistor y cuyos valores se encuentran en la tabla \ref{tab:1-datos:corrientes}.

\begin{table}[H]
    \centering
    \begin{tabular}{@{}rr@{}}
        \toprule
        $R_L$ (nominal, \si{\kilo\ohm}) & $R_L$ (medido, \si{\kilo\ohm})  \\
        \midrule
        \input{src/1/tables/resistencias-l}
    \end{tabular}
    \caption{Valores medidos de $R_L$}
    \label{tab:1-datos:resistencias-l}
\end{table}

\begin{table}[H]
    \centering
    \begin{tabular}{@{}rrrrrr@{}}
        \toprule
        $R_L$ (\si{\kilo\ohm}) & $v_i$ (\si{\volt}) & $v_o$ (\si{\volt}) & 
            $v_+$ (\si{\milli\volt}) & $v_-$ (\si{\milli\volt}) &
            $v_+ - v_-$ (\si{\milli\volt}) \\
        \midrule
        \input{src/1/tables/tensiones-1}
    \end{tabular}
    \caption{Tensiones medidas en el circuito (parte 1)}
    \label{tab:1-datos:tensiones-1}
\end{table}

\begin{table}[H]
    \centering
    \begin{tabular}{@{}rrrrrr@{}}
        \toprule
        $R_L$ (\si{\kilo\ohm}) & $V_{R_1}$ (\si{\volt}) & $V_{R_2}$ (\si{\volt})& $V_{R_3}$ (\si{\milli\volt}) & $V_{R_L}$ (\si{\volt}) \\
        \midrule
        \input{src/1/tables/tensiones-2}
    \end{tabular}
    \caption{Tensiones medidas en el circuito (parte 2)}
    \label{tab:1-datos:tensiones-2}
\end{table}

\begin{table}[H]
    \centering
    \begin{tabular}{@{}rrrrr@{}}
        \toprule
        $R_L$ (\si{\kilo\ohm}) & $I_{R_1}$ (\si{\milli\ampere}) & $I_{R_2}$ (\si{\milli\ampere}) & $I_{R_3}$ (\si{\milli\ampere}) & $I_{R_L}$ (\si{\milli\ampere}) \\
        \midrule
        \input{src/1/tables/corrientes}
    \end{tabular}
    \caption{Corrientes calculadas en el circuito}
    \label{tab:1-datos:corrientes}
\end{table}

La ganancia del operacional puede calcularse experimentalmente como

\begin{equation}
    \label{ec:1-teoria:ganancia-experimental}
    A = \frac{v_o}{v_i}
\end{equation}

\begin{equation}
    \label{ec:1-teoria:err-ganancia-experimental}
    \Delta A = \left| \frac{1}{v_i} \right| \Delta v_o + \left| -\frac{v_o}{{v_i}^2} \right| \Delta v_i
\end{equation}

Los resultados de esta operación se encuentran en la tabla \ref{tab:1-teoria:ganancia-experimental}.


\begin{table}[H]
    \centering
    \begin{tabular}{@{}rr@{}}
        \toprule
        $R_L$ (\si{\kilo\ohm}) & $A$ \\
        \midrule
        \input{src/1/tables/ganancia-experimental}
    \end{tabular}
    \caption{Valores experimentales de ganancia}
    \label{tab:1-teoria:ganancia-experimental}
\end{table}


% \subsection{Análisis de datos}

La fig. \ref{fig:1-analisis:ganancia} compara la ganancia teórica con los valores obtenidos en la práctica.

\begin{figure}[H]
    \includegraphics[width=\textwidth]{img/1-analisis-ganancia.eps}
    \caption{Ganancia teórica vs. ganancia experimental}
    \label{fig:1-analisis:ganancia}
\end{figure}



    \newpage
    \section{Segunda parte: amplificador no inversor}

Se arma el circuito de la fig. \ref{fig:2:esquema} en un \textit{protoboard}
con $R_1 = \SI{10}{\kilo\ohm}$ y $R_2 = \SI{15}{\kilo\ohm}$, un TL072 como
amplificador operacional y una fuente partida de $\SI{14}{\volt}$
alimentándolo, además de una pila de \SI{9}{\volt} como $V_1$.
Luego se utilizan distintos valores de resistencia $R_L$
(47, 10 y \SI{1}{\kilo\ohm}) y se miden las tensiones $v_i$ y $v_o$ para
calcular la ganancia del circuito.

Hecho esto, se modifica el circuito para convertirlo en el seguidor de tensión
de la fig. \ref{fig:2:esquema-seguidor} (ver sección
\ref{sec:intro:opamp-noinversor}) y se vuelven a utilizar los
mismos valores de $R_L$ para hallar los distintos puntos de trabajo.

\begin{figure}[H]
    \centering
    \begin{subfigure}[b]{0.45\textwidth}
        \centering
        \begin{circuitikz}
    \node[op amp] at (0, 0) (opamp) {};

    \draw(opamp.+) -- ++(-0.2, 0) node[label={left:$v_+ = v_i$}]{}
    to[battery1, a=$V_1$] ++(0, -2) coordinate(cg)
    node[ground]{};

    \draw(opamp.-) -- ++(-0.7, 0) node[label={above:$v_-$}]{}
    to[R, a=$R_1$] ++(-1.5, 0)
    to[short] ++(-0.2, 0) coordinate(ci)
    to[short] (cg -| ci) node[ground]{};

    \draw(opamp.-) -- ++(-0.2, 0)
    to[short, *-] ++(0, 2)
    to[R=$R_2$] ++(2.8, 0) coordinate(co)
    to[short] (opamp.out -| co) coordinate(co1) -- (opamp.out);

    \draw(co1) 
    to[short, *-o] ++(0.4, 0) node[right]{$v_o$};

    \draw(co1)
    to[R=$R_L$] (cg -| co1) node[ground]{};

    \draw(opamp.up) -- ++(0, 0.2) node[vcc]{$+V_{cc}$};
    \draw(opamp.down) -- ++(0, -0.2) node[vee]{$-V_{cc}$};
\end{circuitikz}

        \caption{Amplificador no inversor}
        \label{fig:2:esquema}
    \end{subfigure}
    \hfill
    \begin{subfigure}[b]{0.45\textwidth}
        \centering
        \input{src/2/esquematico-seguidor.tikz}
        \caption{Amplificador seguidor de tensión}
        \label{fig:2:esquema-seguidor}
    \end{subfigure}
    \caption{Circuitos a implementar}
\end{figure}

\subsection{Resolución teórica}

El circuito de la fig. \ref{fig:2:esquema} es prácticamente idéntico al de
la fig. \ref{fig:intro:opamp-no-inversor}, con el agregado del resistor $R_L$
que no modifica el valor de la tensión de salida.

Si se utiliza el modelo ideal planteado en la sección
\ref{sec:intro}, se puede despreciar el efecto del resistor
$R_1$ en el circuito de la fig. \ref{fig:2:esquema-seguidor}, ya que no
circulará corriente por esta rama debido a la alta resistencia de entrada del
operacional. De esta forma se obtiene el mismo circuito de la fig.
\ref{fig:intro:opamp-no-inversor}, con resistencia infinita entre la terminal
inversora y masa.

Para ambos casos, entonces, se puede utilizar la ec.
\ref{ec:intro:opamp-noinversor}. Al propagar el error se obtiene nuevamente
la ec. \ref{ec:1-teoria:err-ganancia}, motivo por el cual se obtiene una 
ganancia $A = \num{2.5 \pm 0.15}$ para el caso del amplificador no inversor,
y $A = \num{1.0}$ para el seguidor de tensión (incertidumbre nula, puesto que
$R_1 = \infty$).


\subsection{Simulación con LTSpice}

La simulación de LTSpice de la fig. \ref{fig:2:ltspice} del amplificador no
inversor de la fig. \ref{fig:2:esquema} arrojó una ganancia $A = 2.304$ en
todos los casos, que concuerda con el análisis teórico. El único factor que
varió fue la corriente $I_{R_L}$, de $0.4413$, $2.074$ y
$\SI{20.74}{\milli\ampere}$ para las resistencias de $47$, $10$ y 
$\SI{1}{\kilo\ohm}$.

Para el amplificador seguidor de tensión se simuló el circuito de la fig.
\ref{fig:2:esquema-seguidor} con LTSpice, según lo indicado en la fig.
\ref{fig:2:ltspice-seguidor}. En todos los casos se obtuvo una ganancia $A=0.998$, que
concuerda con lo visto en la sección anterior. La única variable que presentó
diferencias según el valor de $R_L$ que se utilizara fue, por supuesto, la 
corriente por dicho resistor, de $0.1911$, $0.8984$ y
$\SI{8.984}{\milli\ampere}$ para resistencias de $47$, $10$ y
$\SI{1}{\kilo\ohm}$, respectivamente.

\begin{figure}[H]
    \centering
    \includegraphics[width=0.8\textwidth]{img/2/ltspice.png}
    \caption{Simulación en LTSpice del circuito de la fig. \ref{fig:2:esquema}}
    \label{fig:2:ltspice}
\end{figure}

\begin{figure}[H]
    \centering
    \includegraphics[width=0.8\textwidth]{img/2/ltspice-seguidor.png}
    \caption{Simulación en LTSpice del circuito de la fig.
        \ref{fig:2:esquema-seguidor}}
    \label{fig:2:ltspice-seguidor}
\end{figure}


\subsection{Datos obtenidos}

Para el amplificador no inversor, la tensión del nodo de entrada fue
$v_i = \SI{9.61 \pm 0.01}{\volt}$. Para los valores de resistencia 
$R_L$ de 47, 10 y \SI{1}{\kilo\ohm} se obtuvo siempre el mismo valor de
tensión a la salida del operacional: $v_o = \SI{20.90 \pm 0.01}{\volt}$.

Haciendo uso nuevamente de las ecuaciones \ref{ec:1-teoria:ganancia-experimental} y 
\ref{ec:1-teoria:err-ganancia-experimental} para el cálculo de la ganancia y del error, se
obtiene una ganancia $A = \num{2.174817898 \pm 0.003303660664}$.

En el caso del seguidor de tensión, $v_i = \SI{9.59 \pm 0.01}{\volt}$
y la salida registró $v_o = \SI{9.57 \pm 0.01}{\volt}$ para los tres valores
utilizados de $R_L$. La ganancia entonces es 
$A = \num{0.9979144943 \pm 0.002083331068}$.



\subsection{Análisis de datos}

\begin{figure}[H]
    \centering
    \input{img/2/ganancia.tikz}
    \caption{Ganancia del amplificador no inversor}
    \label{fig:2-analisis:ganancia-opamp}
\end{figure}

\begin{figure}[H]
    \centering
    \input{img/2/ganancia-seguidor.tikz}
    \caption{Ganancia del seguidor de tensión}
    \label{fig:2-analisis:ganancia-seguidor}
\end{figure}



    \TODO{Agregar conclusiones}
    
    \newpage
    \section{Bibliografía}
    \printbibliography[heading=none]

\end{document}
