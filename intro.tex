\section{Consigna}

\subsection{Introducción}

A continuación se reproduce la sección de introducción de las consignas de
este trabajo práctico:

El objetivo del trabajo práctico es: que los alumnos puedan implementar una 
solución para un problema del mundo real utilizando las herramientas de algún 
motor de base de datos. El motor en el que se va a efectuar la entrega puede ser
de tipo open source o no. En todos los casos los alumnos deberán asegurarse de 
contar con el software necesario para poder mostrar el trabajo práctico en las 
fechas y lugar de entrega.

Al momento de la corrección se tendrán en cuenta tanto la correctitud de la 
solución como el uso de las herramientas disponibles en el motor elegido.

La entrega deberá constar, como mínimo, de la siguiente documentación:

\begin{itemize}
    \item  Modelo de Entidad Relación y Modelo Relacional derivado, utilizados 
        para implementar la solución.
    \item Detalle de los supuestos asumidos para la resolución del problema.
    \item Diseño físico correspondiente a la solución implementada.
    \item Restricciones adicionales al modelo
    \item Código correspondiente a los stored procedures/ triggers que se hayan 
    implementado en la solución.
\end{itemize}

Además la base que se use para efectuar la demostración deberá contener datos de
prueba cargados de tal forma de poder evaluar la forma en que funcionan las 
consultas que forman parte de los requerimientos.

No es necesario entregar una interfase para ejecutar las consultas, las mismas 
podrán ser ejecutadas directamente desde la interfase del motor de base de datos
elegido.

Recomendamos la consulta periódica con los docentes sobre el avance del trabajo, 
antes de la fecha de entrega.

\subsection{Descripción del problema}

Se desea diseñar una base de datos para controlar la seguridad en un Laboratorio que trabaja con
materiales peligrosos.

Las instalaciones están divididas en áreas, cada una asociada con un nivel de seguridad, que
puede ser baja, media o alta. En un futuro estos niveles podrían modificarse.

El personal de la empresa pertenece a distintas categorías según su función: jerárquico,
profesional y no-profesional, y de todos ellos es necesario registrar sus datos personales.

Cada área está asociada con un nombre, un número de área y con un empleado jerárquico
responsable de la misma. Por otro lado, cada empleado jerárquico tiene asignada un área que
dirige. Es necesario registrar la fecha a partir de la cual ese empleado se hizo cargo del área en
cuestión, y además un historial de los eventos más destacados de dicha área, según fecha y hora
de ocurridos. Por ejemplo, podría agendarse la falta de algún material necesario para cierto
experimento y el reemplazo del mismo.

Los empleados profesionales están asociados con una especialidad y pueden ser de ``planta
permanente'' o ``contratados'' durante un período para desarrollar un trabajo específico. En este
último caso es necesario llevar registro del área para la cual fue contratado el profesional, la tarea
a realizar y el período de contrato. Ocasionalmente, sobre algunos de estos trabajos se realizan
auditorías, siendo necesario registrar fecha y hora de la misma y su resultado. Tener en cuenta que
un empleado de planta trabaja en un área específica, mientras que un profesional contratado pudo
haber trabajado en distintas áreas en caso de haber sido contratado más de una vez.

Por otro lado, los empleados no-profesionales pueden tener acceso a distintas áreas pero éstas
deben pertenecer al mismo nivel de seguridad, ya que la empresa los capacita para ello. Con
respecto a este punto, es necesario saber por cada empleado no-profesional, para qué nivel de
seguridad está preparado, qué áreas tiene asignadas actualmente y en qué franja horaria está
autorizado su acceso por área. Las franjas horarias o turnos están estandarizados dentro del
Laboratorio, por ejemplo, 8 a 12hs, 12 a 16hs, etc.

Las áreas con seguridad media y alta tienen acceso restringido. Cada vez que un empleado ingresa
o abandona un área, el mismo debe registrarse en el sistema a través de su huella dactilar. En caso
de no funcionar este sistema (por ejemplo, no funciona el lector de huellas), el ingreso/egreso se
hará a través de su número identificatorio y una contraseña. El sistema registra, entonces, fecha y
hora de ingreso/egreso de cada empleado a un área específica y si la acción fue autorizada o no.
Todo empleado (incluyendo profesionales contratados), cuenta con un número identificatorio
dentro del Laboratorio.

Las funcionalidades que se esperan implementadas (stored procedures/triggers) son:

\begin{itemize}
    \item Obtener el nombre, apellido y número identificatorio de los empleados No Profesionales
que pueden ingresar a todas las áreas del nivel de seguridad asignado.
    \item Mediante una vista obtener los empleados que en el día de la fecha han realizado algún
intento de ingreso fallido a un área sin contar con un ingreso exitoso posterior para la
misma. Incluir el área donde intento ingresar en las columnas que devuelve la vista.
    \item Obtener los datos personales de los empleados que en los últimos 30 días cuentan con una
cantidad de intentos fallidos mayor a 5 o con al menos un intento de ingreso en un área
cuyo nivel de seguridad sea superior al que tienen asignado.
    \item Implementar un control en la base de datos que impida que a un empleado se le asigne un
área si no está capacitado para el nivel de seguridad de ese área.
    \item Implementación de alguna restricción adicional que surja del diseño.
    \item Desarrollar una aplicación sencilla en un lenguaje host que acceda a la base de datos y
permita realizar una operación sobre la misma, justificando el uso de transacciones
\end{itemize}
