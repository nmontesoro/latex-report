\subsection{Resolución teórica}

Considerando al operacional como un elemento ideal (ver sección \ref{sec:intro}) se sabe que su resistencia de entrada es infinita. Por tal motivo no circulará corriente a través de la resistencia $R_3$, lo cual implica que no habrá caída de tensión alguna entre sus terminales. Efectivamente es como tener el circuito descrito en la sección \ref{sec:intro:opamp-inversor} y por ende se puede usar la ecuación \ref{ec:intro:opamp-inversor}. Entonces queda:

\begin{equation}
    \label{ec:1-teoria:ganancia}
    A = -\frac{R_2}{R_1}
\end{equation}

\begin{equation}
    \label{ec:1-teoria:err-ganancia}
    \Delta A = \left| - \frac{1}{R_1} \right| \Delta R_2
             + \left| - \frac{R_2}{{R_1}^2} \right| \Delta R_1
\end{equation}

