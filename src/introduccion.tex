\section{Introducción}

El período $T$ de un péndulo físico está dado por la ecuación
\ref{ec:intro:periodo}, donde $L$ es la longitud del péndulo y $g$ es la
constante de la gravedad:

\begin{equation}
    \label{ec:intro:periodo}
    T = 2\pi \sqrt{\frac{2L}{3g}}
\end{equation}

De la ecuación anterior puede despejarse $g$, de forma tal de poder calcular el
valor de esta constante (y su incerteza) midiendo la longitud y el tiempo que
tarda un péndulo en completar una oscilación:

\begin{equation}
    \label{ec:intro:gravedad}
    g = \frac{8}{3} \pi^2 \frac{L}{T^2}
\end{equation}

\begin{equation}
    \label{ec:intro:error-gravedad}
    \Delta g = \frac{8}{3} \pi^2 \left( \frac{\Delta L}{T^2} +
                                        \frac{2L \Delta T}{T^3}
                                    \right)
\end{equation}
donde $\Delta L$ y $\Delta T$ son las incertidumbres en las mediciones de la
longitud y el período del péndulo, respectivamente. 
