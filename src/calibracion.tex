\subsection{Curva de calibración}
\label{sec:calibracion}

La capacidad de un arreglo de dos capacitores en paralelo es la suma de las
capacidades de cada capacitor, y la propagación del error de esta expresión 
resulta en la suma de las incertidumbres en las mediciones de capacidad de cada
capacitor.

La capacidad total de un arreglo de capacitores en serie está dada por la ec.
\ref{ec:calibracion:paralelo}, y el error en esta expresión se calcula con la
ec. \ref{ec:calibracion:error-paralelo}:

\vspace{10mm}
\begin{equation}
    \label{ec:calibracion:paralelo}
    C = \frac{1}{\frac{1}{C_1} + \frac{1}{C_2}}
\end{equation}
\vspace{10mm}
\begin{equation}
    \label{ec:calibracion:error-paralelo}
    \Delta C =
    \frac{1}{{C_1}^2 \left( \frac{1}{C_1} + \frac{1}{C_2} \right)^2} \, \Delta C_1 + 
    \frac{1}{{C_2}^2 \left( \frac{1}{C_1} + \frac{1}{C_2} \right)^2} \, \Delta C_2
\end{equation}
\vspace{10mm}

Se utilizaron 6 configuraciones distintas. Los resultados de las mediciones se
encuentran en la tabla \ref{tab:calibracion:datos}.
El error en $f$ fue calculado según la expresión
$\Delta f = 3\%\,\text{rdg} + 5\,\text{dgt}
          = 0.3\, f + \SI{50}{\hertz}$.

\vspace{10mm}
\begin{table}[H]
    \centering
    \csvreader[
        no head,
        tabular = r|rr,
        table head = \toprule Configuración & $C$ (\si{\pico\farad}) & $f$ (\si{\kilo\hertz}) \\\midrule,
        table foot = \bottomrule,
    ]{data/calibracion.csv}{}{%
        $\csvcoli$ & \num{\csvcolii \pm \csvcoliii} & \num{\csvcoliv \pm \csvcolv}
    }
    \caption{Resultados de mediciones para curva de calibración}
    \label{tab:calibracion:datos}
\end{table}
\vspace{10mm}

Con los datos de la tabla \ref{tab:calibracion:datos} se consigue el gráfico
de la fig. \ref{fig:calibracion:rectas}, en el cual se superpone la recta
de mejor ajuste obtenida mediante la construcción de las rectas de mínima y 
máxima pendiente. La región sombreada indica las rectas que sería posible 
construir considerando los errores en $m$ y en $b$.

La ecuación de la recta mencionada es $f^{-1} = mC + b$,
donde \csvreader[no head]{data/calculos-1.csv}{}{
    $m = \SI{\csvcolvi \pm \csvcolviii}{\per\pico\farad\kilo\hertz}$ y
    $b = \SI{\csvcolvii \pm \csvcolix}{\per\kilo\hertz}$.
}

\begin{figure}[H]
    \centering
    \subsection{Curva de calibración}
\label{sec:calibracion}

La capacidad de un arreglo de dos capacitores en paralelo es la suma de las
capacidades de cada capacitor, y la propagación del error de esta expresión 
resulta en la suma de las incertidumbres en las mediciones de capacidad de cada
capacitor.

La capacidad total de un arreglo de capacitores en serie está dada por la ec.
\ref{ec:calibracion:paralelo}, y el error en esta expresión se calcula con la
ec. \ref{ec:calibracion:error-paralelo}:

\vspace{10mm}
\begin{equation}
    \label{ec:calibracion:paralelo}
    C = \frac{1}{\frac{1}{C_1} + \frac{1}{C_2}}
\end{equation}
\vspace{10mm}
\begin{equation}
    \label{ec:calibracion:error-paralelo}
    \Delta C =
    \frac{1}{{C_1}^2 \left( \frac{1}{C_1} + \frac{1}{C_2} \right)^2} \, \Delta C_1 + 
    \frac{1}{{C_2}^2 \left( \frac{1}{C_1} + \frac{1}{C_2} \right)^2} \, \Delta C_2
\end{equation}
\vspace{10mm}

Se utilizaron 6 configuraciones distintas. Los resultados de las mediciones se
encuentran en la tabla \ref{tab:calibracion:datos}.

\vspace{10mm}
\begin{table}[H]
    \centering
    \csvreader[
        no head,
        tabular = r|rr,
        table head = \toprule Configuración & $C$ (\si{\pico\farad}) & $f$ (\si{\kilo\hertz}) \\\midrule,
        table foot = \bottomrule,
    ]{data/calibracion.csv}{}{%
        $\csvcoli$ & \num{\csvcolii \pm \csvcoliii} & \num{\csvcoliv \pm \csvcolv}
    }
    \caption{Resultados de mediciones para curva de calibración}
    \label{tab:calibracion:datos}
\end{table}
\vspace{10mm}

Con los datos de la tabla \ref{tab:calibracion:datos} se consigue el gráfico
de la fig. \ref{fig:calibracion:rectas}, en el cual se superpone la recta
de mejor ajuste obtenida mediante la construcción de las rectas de mínima y 
máxima pendiente. La región sombreada indica el error en $m$ y en $b$.

La ecuación de la recta mencionada es $f^{-1} = mC + b$,
donde \csvreader[no head]{data/calculos-1.csv}{}{
    $m = \SI{\csvcolvi \pm \csvcolviii}{\pico\farad\kilo\hertz}$ y
    $b = \SI{\csvcolvii \pm \csvcolix}{\per\kilo\hertz}$.
}

\begin{figure}[H]
    \centering
    \subsection{Curva de calibración}
\label{sec:calibracion}

La capacidad de un arreglo de dos capacitores en paralelo es la suma de las
capacidades de cada capacitor, y la propagación del error de esta expresión 
resulta en la suma de las incertidumbres en las mediciones de capacidad de cada
capacitor.

La capacidad total de un arreglo de capacitores en serie está dada por la ec.
\ref{ec:calibracion:paralelo}, y el error en esta expresión se calcula con la
ec. \ref{ec:calibracion:error-paralelo}:

\vspace{10mm}
\begin{equation}
    \label{ec:calibracion:paralelo}
    C = \frac{1}{\frac{1}{C_1} + \frac{1}{C_2}}
\end{equation}
\vspace{10mm}
\begin{equation}
    \label{ec:calibracion:error-paralelo}
    \Delta C =
    \frac{1}{{C_1}^2 \left( \frac{1}{C_1} + \frac{1}{C_2} \right)^2} \, \Delta C_1 + 
    \frac{1}{{C_2}^2 \left( \frac{1}{C_1} + \frac{1}{C_2} \right)^2} \, \Delta C_2
\end{equation}
\vspace{10mm}

Se utilizaron 6 configuraciones distintas. Los resultados de las mediciones se
encuentran en la tabla \ref{tab:calibracion:datos}.

\vspace{10mm}
\begin{table}[H]
    \centering
    \csvreader[
        no head,
        tabular = r|rr,
        table head = \toprule Configuración & $C$ (\si{\pico\farad}) & $f$ (\si{\kilo\hertz}) \\\midrule,
        table foot = \bottomrule,
    ]{data/calibracion.csv}{}{%
        $\csvcoli$ & \num{\csvcolii \pm \csvcoliii} & \num{\csvcoliv \pm \csvcolv}
    }
    \caption{Resultados de mediciones para curva de calibración}
    \label{tab:calibracion:datos}
\end{table}
\vspace{10mm}

Con los datos de la tabla \ref{tab:calibracion:datos} se consigue el gráfico
de la fig. \ref{fig:calibracion:rectas}, en el cual se superpone la recta
de mejor ajuste obtenida mediante la construcción de las rectas de mínima y 
máxima pendiente. La región sombreada indica el error en $m$ y en $b$.

La ecuación de la recta mencionada es $f^{-1} = mC + b$,
donde \csvreader[no head]{data/calculos-1.csv}{}{
    $m = \SI{\csvcolvi \pm \csvcolviii}{\pico\farad\kilo\hertz}$ y
    $b = \SI{\csvcolvii \pm \csvcolix}{\per\kilo\hertz}$.
}

\begin{figure}[H]
    \centering
    \subsection{Curva de calibración}
\label{sec:calibracion}

La capacidad de un arreglo de dos capacitores en paralelo es la suma de las
capacidades de cada capacitor, y la propagación del error de esta expresión 
resulta en la suma de las incertidumbres en las mediciones de capacidad de cada
capacitor.

La capacidad total de un arreglo de capacitores en serie está dada por la ec.
\ref{ec:calibracion:paralelo}, y el error en esta expresión se calcula con la
ec. \ref{ec:calibracion:error-paralelo}:

\vspace{10mm}
\begin{equation}
    \label{ec:calibracion:paralelo}
    C = \frac{1}{\frac{1}{C_1} + \frac{1}{C_2}}
\end{equation}
\vspace{10mm}
\begin{equation}
    \label{ec:calibracion:error-paralelo}
    \Delta C =
    \frac{1}{{C_1}^2 \left( \frac{1}{C_1} + \frac{1}{C_2} \right)^2} \, \Delta C_1 + 
    \frac{1}{{C_2}^2 \left( \frac{1}{C_1} + \frac{1}{C_2} \right)^2} \, \Delta C_2
\end{equation}
\vspace{10mm}

Se utilizaron 6 configuraciones distintas. Los resultados de las mediciones se
encuentran en la tabla \ref{tab:calibracion:datos}.

\vspace{10mm}
\begin{table}[H]
    \centering
    \csvreader[
        no head,
        tabular = r|rr,
        table head = \toprule Configuración & $C$ (\si{\pico\farad}) & $f$ (\si{\kilo\hertz}) \\\midrule,
        table foot = \bottomrule,
    ]{data/calibracion.csv}{}{%
        $\csvcoli$ & \num{\csvcolii \pm \csvcoliii} & \num{\csvcoliv \pm \csvcolv}
    }
    \caption{Resultados de mediciones para curva de calibración}
    \label{tab:calibracion:datos}
\end{table}
\vspace{10mm}

Con los datos de la tabla \ref{tab:calibracion:datos} se consigue el gráfico
de la fig. \ref{fig:calibracion:rectas}, en el cual se superpone la recta
de mejor ajuste obtenida mediante la construcción de las rectas de mínima y 
máxima pendiente. La región sombreada indica el error en $m$ y en $b$.

La ecuación de la recta mencionada es $f^{-1} = mC + b$,
donde \csvreader[no head]{data/calculos-1.csv}{}{
    $m = \SI{\csvcolvi \pm \csvcolviii}{\pico\farad\kilo\hertz}$ y
    $b = \SI{\csvcolvii \pm \csvcolix}{\per\kilo\hertz}$.
}

\begin{figure}[H]
    \centering
    \input{img/calibracion.tikz}
    \caption{Gráfico de calibración}
    \label{fig:calibracion:rectas}
\end{figure}

Despejando la ecuación de transformación en frecuencias
(\ref{ec:calibracion:transformacion}) se obtiene una expresión compatible con
la forma de una recta, ec. \ref{ec:calibracion:transformacion-recta}:

\vspace{10mm}
\begin{equation}
    \label{ec:calibracion:transformacion}
    f = \frac{\alpha}{C + C_o}
\end{equation}
\vspace{10mm}
\begin{equation}
    \label{ec:calibracion:transformacion-recta}
    \frac{1}{f} = \frac{1}{\alpha} C + \frac{C_o}{\alpha}
\end{equation}
\vspace{10mm}

Con la ecuación \ref{ec:calibracion:transformacion-recta} se puede plantear que

\vspace{5mm}
\[
    \alpha = m^{-1} \pm \left| -m^{-2} \right| \Delta m,
    \quad\quad
    C_o = bm^{-1} \pm \left( \left| m^{-1} \right| \Delta b +
        \left| -bm^{-2} \right| \Delta m \right)
\]
\vspace{5mm}

Con lo que se consiguen los valores de \csvreader[no head]{data/calculos-1.csv}{}{
    $\alpha = \SI{\csvcolx \pm \csvcolxi}{\per\pico\farad\kilo\hertz}$ y 
    $C_o = \SI{\csvcolxii \pm \csvcolxiii}{\pico\farad}$.
}

    \caption{Gráfico de calibración}
    \label{fig:calibracion:rectas}
\end{figure}

Despejando la ecuación de transformación en frecuencias
(\ref{ec:calibracion:transformacion}) se obtiene una expresión compatible con
la forma de una recta, ec. \ref{ec:calibracion:transformacion-recta}:

\vspace{10mm}
\begin{equation}
    \label{ec:calibracion:transformacion}
    f = \frac{\alpha}{C + C_o}
\end{equation}
\vspace{10mm}
\begin{equation}
    \label{ec:calibracion:transformacion-recta}
    \frac{1}{f} = \frac{1}{\alpha} C + \frac{C_o}{\alpha}
\end{equation}
\vspace{10mm}

Con la ecuación \ref{ec:calibracion:transformacion-recta} se puede plantear que

\vspace{5mm}
\[
    \alpha = m^{-1} \pm \left| -m^{-2} \right| \Delta m,
    \quad\quad
    C_o = bm^{-1} \pm \left( \left| m^{-1} \right| \Delta b +
        \left| -bm^{-2} \right| \Delta m \right)
\]
\vspace{5mm}

Con lo que se consiguen los valores de \csvreader[no head]{data/calculos-1.csv}{}{
    $\alpha = \SI{\csvcolx \pm \csvcolxi}{\per\pico\farad\kilo\hertz}$ y 
    $C_o = \SI{\csvcolxii \pm \csvcolxiii}{\pico\farad}$.
}

    \caption{Gráfico de calibración}
    \label{fig:calibracion:rectas}
\end{figure}

Despejando la ecuación de transformación en frecuencias
(\ref{ec:calibracion:transformacion}) se obtiene una expresión compatible con
la forma de una recta, ec. \ref{ec:calibracion:transformacion-recta}:

\vspace{10mm}
\begin{equation}
    \label{ec:calibracion:transformacion}
    f = \frac{\alpha}{C + C_o}
\end{equation}
\vspace{10mm}
\begin{equation}
    \label{ec:calibracion:transformacion-recta}
    \frac{1}{f} = \frac{1}{\alpha} C + \frac{C_o}{\alpha}
\end{equation}
\vspace{10mm}

Con la ecuación \ref{ec:calibracion:transformacion-recta} se puede plantear que

\vspace{5mm}
\[
    \alpha = m^{-1} \pm \left| -m^{-2} \right| \Delta m,
    \quad\quad
    C_o = bm^{-1} \pm \left( \left| m^{-1} \right| \Delta b +
        \left| -bm^{-2} \right| \Delta m \right)
\]
\vspace{5mm}

Con lo que se consiguen los valores de \csvreader[no head]{data/calculos-1.csv}{}{
    $\alpha = \SI{\csvcolx \pm \csvcolxi}{\per\pico\farad\kilo\hertz}$ y 
    $C_o = \SI{\csvcolxii \pm \csvcolxiii}{\pico\farad}$.
}

    \caption{Gráfico de calibración}
    \label{fig:calibracion:rectas}
\end{figure}

Despejando la ecuación de transformación en frecuencias
(\ref{ec:calibracion:transformacion}) se obtiene una expresión compatible con
la forma de una recta, ec. \ref{ec:calibracion:transformacion-recta}:

\vspace{10mm}
\begin{equation}
    \label{ec:calibracion:transformacion}
    f = \frac{\alpha}{C + C_o}
\end{equation}
\vspace{10mm}
\begin{equation}
    \label{ec:calibracion:transformacion-recta}
    \frac{1}{f} = \frac{1}{\alpha} C + \frac{C_o}{\alpha}
\end{equation}
\vspace{10mm}

Con la ecuación \ref{ec:calibracion:transformacion-recta} se puede plantear que

\vspace{5mm}
\[
    \alpha = m^{-1} \pm \left| -m^{-2} \right| \Delta m,
    \quad\quad
    C_o = bm^{-1} \pm \left( \left| m^{-1} \right| \Delta b +
        \left| -bm^{-2} \right| \Delta m \right)
\]
\vspace{5mm}

Con lo que se consiguen los valores de \csvreader[no head]{data/calculos-1.csv}{}{
    $\alpha = \SI{\csvcolx \pm \csvcolxi}{\pico\farad\kilo\hertz}$ y 
    $C_o = \SI{\csvcolxii \pm \csvcolxiii}{\pico\farad}$.
}
