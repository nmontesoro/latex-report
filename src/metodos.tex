\section{Métodos y procedimiento}

Los elementos a utilizar son:

\begin{itemize}
    \item Una placa de experimentación o \textit{protoboard}
    \item Un circuito integrado TL072
    \item Resistores de \SI{1}{\kilo\ohm}, \SI{10}{\kilo\ohm},
        \SI{15}{\kilo\ohm} y \SI{47}{\kilo\ohm} (tres de cada valor)
    \item Una batería de \SI{9}{\volt} con clip para conectarla
    \item Multímetro digital
    \item Osciloscopio digital
    \item Generador de funciones
    \item Fuente de alimentación de laboratorio
    \item Alicate
    \item Cables para \textit{protoboard}
\end{itemize}

El procedimiento para ambos casos consiste en montar los circuitos según su 
correspondiente esquemático en una \textit{protoboard} para medir las tensiones
de entrada y de salida del operacional, las tensiones en las terminales inversora y
no inversora (y las diferencias de potencial entre éstas) y las corrientes en
cada resistencia, haciendo uso del multímetro configurado en las escalas
correspondientes.
