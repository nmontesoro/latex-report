\section{Métodos y procedimiento}

Los elementos a utilizar son:

\begin{itemize}
    \item Una placa de experimentación o \textit{protoboard}
    \item Un circuito integrado LM358\footnote{Se utilizó un TL072 en su lugar.}
    \item Resistores de \SI{1}{\kilo\ohm}, \SI{10}{\kilo\ohm},
        \SI{15}{\kilo\ohm} y \SI{47}{\kilo\ohm} (tres de cada uno)
    \item Dos pilas de \SI{1.5}{\volt} con portapilas
    \item Una batería de \SI{9}{\volt} con clip para conectarla
    \item Multímetro
    \item Osciloscopio
    \item Generador de funciones
    \item Fuente de alimentación de laboratorio
\end{itemize}

El procedimiento para ambos casos consiste en armar los circuitos según su 
correspondiente esquemático en una \textit{protoboard} para medir las tensiones
de entrada y salida del operacional, las tensiones en las terminales inversora y
no inversora (y las diferencias de potencial entre éstas) y las corrientes en
cada resistencia, haciendo uso del multímetro configurado en la escala que
corresponda.
