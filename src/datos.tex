\section{Análisis de datos}

Los valores de período obtenidos experimentalmente se encuentran resumidos en
la fig. \ref{fig:datos:periodos}.

\begin{figure}[H]
    \centering
    \input{img/periodos.tikz}
    \caption{Períodos obtenidos}
    \label{fig:datos:periodos}
\end{figure}

Se decide utilizar un gráfico $T$ vs. $\sqrt{L}$ por las razones que se 
exponen en el apéndice de este trabajo (sección
\ref{sec:apendice:justificacion}). Mediante mínimos cuadrados se 
encuentra que la recta que mejor se adapta a los datos experimentales es

\[
    y_i = Bx_i + A, \quad\quad x_i = \sqrt{L_i}, \quad y_i = T_i
\]
habiendo calculado los siguientes valores (ver apéndice, sección 
\ref{sec:apendice:formulas-estimadores-mc}):

\begin{itemize}
    \input{src/datos/datos-mc}
\end{itemize}

\comment{Lo siguiente podría ir en alguna otra sección. Tal vez en
    ``Discusión''?}

Al aplicar $\lim_{N\to\infty}$ a las fórmulas de la sección
\ref{sec:apendice:formulas-estimadores-mc} se determina que la ordenada al
origen $A$ y las desviaciones $\sigma_{y_i}$, $\sigma_B$ y $\sigma_A$ tenderán
a cero conforme se aumenta la cantidad de mediciones (y, por consiguiente,
también ocurrirá lo mismo con los errores $\Delta B$ y $\Delta A$). Por otro
lado, el valor de la pendiente convergerá a

\[
    B = \frac{ \sum{x_i y_i} }{ \sum {x_i}^2 }
\]

Es decir: al aumentar la cantidad de puntos que se miden, las desviaciones se
anulan y la pendiente debería converger al valor teórico dado por la ec.
\ref{ec:intro:gravedad}, si el modelo es adecuado:

\[
    B = 2\pi \sqrt{\frac{2}{3g}}
\]

Por lo anterior es lógico esperar que el valor de $g$ también tienda a su valor
verdadero.

\comment{Hasta acá.}

\begin{figure}[H]
    \centering
    \input{img/cuadrados.tikz}
    \caption{Mínimos cuadrados}
    \label{fig:datos:regresion}
\end{figure}

Teniendo la recta de la fig. \ref{fig:datos:regresion} se pueden construir los
gráficos de las figs. \ref{fig:datos:residuos} y \ref{fig:datos:histograma}.
La cantidad de barras del histograma fue determinada mediante la regla de 
Sturges \cite{sturges}.
A simple vista parecen cumplirse las condiciones de cuadrados mínimos 
\cite[pág. 527]{estadistica}, que establecen que, dado un modelo lineal
$y_i = A + Bx_i + \varepsilon_i$ con errores $\varepsilon_i$:

\begin{itemize}
    \item Los errores son aleatorios e independientes. La magnitud de 
        cualquiera de ellos no influye en el valor del siguiente.
    \item Todos los errores tienen media 0.
    \item Todos los errores tienen la misma varianza.
    \item Todos los errores están distribuidos normalmente.
\end{itemize}

Viendo la fig. \ref{fig:datos:residuos} parecería que la dispersión vertical
del diagrama no varía demasiado, que el diagrama es homoscedástico, por ende
es \textit{probable} que los supuestos del modelo lineal sean válidos
\cite[págs. 527-528]{estadistica}. Además, en la fig.
\ref{fig:datos:histograma} podemos observar que el grafico se aproxima al
de una campana gaussiana. Esto coincide con lo previsto ya que, para poder
aplicar un ajuste por cuadrados mínimos, es condición necesaria que cada
error en el eje $y$ responda a una distribución gaussiana. Por otro lado la
media $\mu$ de los datos del histograma parece ser muy cercana a cero, también cumpliendo una de las condiciones mencionadas.

\begin{figure}[H]
    \centering
    \input{img/residuos.tikz}
    \caption{residuos}
    \label{fig:datos:residuos}
\end{figure}

\begin{figure}[H]
    \centering
    \input{img/histograma.tikz}
    \caption{histograma de residuos}
    \label{fig:datos:histograma}
\end{figure}

A partir de las ecuaciones \ref{ec:intro:gravedad} y
\ref{ec:intro:error-gravedad} de la sección de Introducción, se
obtienen

\[
    g = \frac{8\pi^2}{3B^2}, \quad \Delta g = \frac{16\pi^2}{3B^3} \Delta B
\]

Es así que \input{src/datos/gravedad.tex}.
