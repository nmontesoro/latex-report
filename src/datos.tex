\section{Datos obtenidos}

Para obtener la densidad $\delta$ de la arena se utilizó un volumen de arena
conocido junto con su masa. Estas medidas se repitieron un total de cuatro 
veces, para intentar minimizar la incertidumbre en la medición. Los resultados
de las mediciones se encuentran en la tabla \ref{tab:datos:densidades}, y
se encontró un valor final de
\csvreader[]{data/densidad.csv}{}{
    $\delta = \bar{\delta} \pm 1.5\,\sigma_{\delta} = 
    \SI{\csvcoli \pm \csvcolii}{\gram\per\milli\liter}$}.

\begin{table}[H]
    \centering
    \csvreader[
        tabular = ccc,
        table head = \toprule $m$ (\si{\gram} $\pm\, 0.1$) & $V$ (\si{\milli\liter} $\pm\, 0.5$) & $\delta$ (\si{\gram\per\milli\liter}) \\\midrule,
        table foot = \bottomrule,
    ]{data/mediciones-densidad.csv}{}{
        \num{\csvcoli} & \num{\csvcolii} & \num[round-mode = places, round-precision = 2]{\csvcoliii}
    }
    \caption{Datos obtenidos para la densidad}
    \label{tab:datos:densidades}
\end{table}

Con un calibre se midieron cuatro veces los diámetros de los orificios de cada 
tapa, para minimizar la incertidumbre de las mediciones teniendo en cuenta
que los bordes de estos orificios no eran perfectos. Los resultados de estas 
mediciones se encuentran en la tabla \ref{tab:datos:mediciones-diametros}.
Tomando $d = \bar{d} \pm 1.5\, \sigma_d$ se calculan los valores de la tabla
\ref{tab:datos:diametros} que se utilizarán en los cálculos.

\begin{table}[H]
    \centering
    \csvreader[
        tabular = r|cccc,
        table head = \toprule Número & Tapa 1 & Tapa 2 & Tapa 3 & Tapa 4 \\\midrule,
        table foot = \bottomrule,
    ]{data/mediciones-diametros.csv}{}{
        \csvcoli & \num{\csvcolii} & \num{\csvcoliii} & \num{\csvcoliv} & \num{\csvcolv}
    }
    \caption{Diámetros medidos de cada tapa. Valores expresados en \si{\mm}
    ($\pm\, 0.02$).}
    \label{tab:datos:mediciones-diametros}
\end{table}

\begin{table}[H]
    \centering
    \csvreader[
        tabular = r|c,
        table head = \toprule Tapa & $d$ (\si{\mm}) \\\midrule,
        table foot = \bottomrule,
    ]{data/diametros.csv}{}{
        \csvcoli & \num{\csvcolii \pm \csvcoliii}
    }
    \caption{Diámetros calculados para cada tapa.}
    \label{tab:datos:diametros}
\end{table}

Para cada tamaño de orificio se midieron cuatro veces distintos parámetros de la
recta $F$ (\si{\newton}) vs. $t$ (\si{\second}) con la ayuda de un software
informático. Los resultados se encuentran en la tabla 
\ref{tab:datos:mediciones-pendientes}.

\begin{table}[H]
    \centering
    \csvreader[
        tabular = r|cc|cc|c,
        table head = \toprule Tapa & $m$ (\si{\newton\per\second}) & $\sigma_m$ (\si{\newton\per\second}) & $b$ (\si{\newton}) & $\sigma_b$ (\si{\newton}) & $r$ \\\midrule,
        table foot = \bottomrule,
    ]{data/mediciones-pendientes.csv}{}{
        \csvcoli & \num{\csvcolii} & \num{\csvcolv} & \num{\csvcoliii} & \num{\csvcolvi} & \num{\csvcoliv}
    }
    \caption{Valores medidos de pendientes y ordenadas.}
    \label{tab:datos:mediciones-pendientes}
\end{table}
