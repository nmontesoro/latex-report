\section{Análisis de datos}

\comment{
    Las figuras \ref{fig:datos:periodos} y \ref{fig:datos:regresion},
    ¿deberían tener barras de error?
}

Los valores de período obtenidos experimentalmente se encuentran resumidos en
la fig. \ref{fig:datos:periodos}.

\begin{figure}[H]
    \centering
    \input{img/periodos.tikz}
    \caption{Períodos obtenidos}
    \label{fig:datos:periodos}
\end{figure}

Se decide utilizar un gráfico $T$ vs. $\sqrt{L}$ por las razones que se 
exponen en el apéndice de este trabajo. Mediante mínimos cuadrados se encuentra
que la recta que mejor se adapta a los datos experimentales es

\[
    y_i = Bx_i + A, \quad\quad x_i = \sqrt{L_i}, \quad y_i = T_i
\]
donde la pendiente es \input{src/datos/pendiente} y la ordenada
\input{src/datos/ordenada}. La fig. \ref{fig:datos:regresion} ilustra tanto
los datos como esta recta.

\begin{figure}[H]
    \centering
    \input{img/cuadrados.tikz}
    \caption{Mínimos cuadrados}
    \label{fig:datos:regresion}
\end{figure}

Teniendo la recta de la fig. \ref{fig:datos:regresion} se pueden construir los
gráficos de las figs. \ref{fig:datos:residuos} y \ref{fig:datos:histograma}.
A simple vista parecen cumplirse las condiciones de cuadrados mínimos 
\cite[pág. 527]{estadistica}, que establecen que, dado un modelo lineal
$y_i = A + Bx_i + \varepsilon_i$ con errores $\varepsilon_i$:

\begin{itemize}
    \item Los errores son aleatorios e independientes. La magnitud de 
        cualquiera de ellos no influye en el valor del siguiente.
    \item Todos los errores tienen media 0.
    \item Todos los errores tienen la misma varianza.
    \item Todos los errores están distribuidos normalmente.
\end{itemize}

Viendo la fig. \ref{fig:datos:residuos} parecería que la dispersión vertical
del diagrama no varía demasiado, que el diagrama es homoscedástico, por ende
es \textit{probable} que los supuestos del modelo lineal sean válidos
\cite[págs. 527-528]{estadistica}. Además, en la fig.
\ref{fig:datos:histograma} se observa claramente la distribución normal de 
los errores, la última de las condiciones mencionadas antes.


\begin{figure}[H]
    \centering
    \input{img/residuos.tikz}
    \caption{residuos}
    \label{fig:datos:residuos}
\end{figure}

\begin{figure}[H]
    \centering
    \input{img/histograma.tikz}
    \caption{histograma de residuos}
    \label{fig:datos:histograma}
\end{figure}
