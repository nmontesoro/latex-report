\section{Análisis de datos}

\comment{
    Las figuras \ref{fig:datos:periodos} y \ref{fig:datos:regresion},
    ¿deberían tener barras de error?
}

Los valores de período obtenidos experimentalmente se encuentran resumidos en
la fig. \ref{fig:datos:periodos}.

\begin{figure}[H]
    \centering
    \input{img/periodos.tikz}
    \caption{Períodos obtenidos}
    \label{fig:datos:periodos}
\end{figure}

Se decide utilizar un gráfico $T$ vs. $\sqrt{L}$ por las razones que se 
exponen en el apéndice de este trabajo. Mediante mínimos cuadrados se encuentra
que la recta que mejor se adapta a los datos experimentales es

\[
    y_i = Bx_i + A, \quad\quad x_i = \sqrt{L_i}, \quad y_i = T_i
\]
donde la pendiente es \input{src/datos/pendiente} y la ordenada
\input{src/datos/ordenada}. La fig. \ref{fig:datos:regresion} ilustra tanto
los datos como esta recta.

\begin{figure}[H]
    \centering
    \input{img/cuadrados.tikz}
    \caption{Mínimos cuadrados}
    \label{fig:datos:regresion}
\end{figure}

\begin{figure}[H]
    \centering
    \input{img/residuos.tikz}
    \caption{residuos}
    \label{fig:datos:residuos}
\end{figure}

\begin{figure}[H]
    \centering
    \input{img/histograma.tikz}
    \caption{histograma de residuos}
    \label{fig:datos:histograma}
\end{figure}
