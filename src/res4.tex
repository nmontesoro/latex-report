\question{
Si el valor de la potencia del diámetro no coincide con el predicho por el 
análisis dimensional, usted puede tener en cuenta lo siguiente: es razonable
asumir que el flujo está afectado por el choque de los granos individuales de 
arena y los bordes del orificio de salida. Debido al tamaño finito de los 
granos individuales, es necesario modificar la ecuación. Entonces:

\begin{equation*}
    \odv{m}{t} = Q = F \left( \delta,\, g,\, d - \varepsilon \right)
\end{equation*}

Donde $\varepsilon$ es la magnitud de la reducción del diámetro. En una 
primera aproximación, parece razonable elegir $\varepsilon$ igual al tamaño
típico del grano de arena.

Proponga esta corrección y determine de sus valores experimentales el valor
de $\varepsilon$ con su error.

}

\answer{

Se reescribe la ecuación obtenida en el primer ejercicio como la ec. 
\ref{ec:4:caudal}. Luego de elevar ambos lados de la igualdad a la 
$2/5$ se obtiene la ecuación \ref{ec:4:caudal-elevado}.

\begin{equation}
    \label{ec:4:caudal}
    C\left(d\right) = K\, \delta\, g^{1/2}\, \left( d - \varepsilon \right) ^ {5/2}
\end{equation}

\begin{equation}
    \label{ec:4:caudal-elevado}
    C^{2/5} = B \, \left( d - \varepsilon \right),\ 
    B = \left[ k\, \delta\, g^{1/2} \right] ^ {2/5}
\end{equation}

Entonces, si se elevan los valores de caudal de la tabla \ref{tab:2:caudales} a la
$2/5$ y se vuelve a realizar una regresión lineal, se obtendrá como resultado 
una recta de mejor ajuste de pendiente $B$ y ordenada $A = -B\varepsilon$. Por
lo tanto, repitiendo los cálculos del ejercicio 3 se obtienen los valores de 
la tabla \ref{tab:4:regresion}:

\begin{table}[H]
    \centering
    \csvreader[
        table head = \toprule $B$ & $A$ \\\midrule,
        table foot = \bottomrule,
        tabular = cc
    ]{data/regresion2.csv}{}{
        \num{\csvcolii \pm \csvcoliv} & \num{\csvcoli \pm \csvcoliii}
    }
    \caption{Valores de regresión para $C^{2/5}$ con modelo ajustado.}
    \label{tab:4:regresion}
\end{table}

Con los valores de la tabla \ref{tab:4:regresion} se calcula un valor 
$\varepsilon$ de
\csvreader[]{data/regresion2.csv}{}{\SI{\csvcolv \pm \csvcolvi}{\mm}}. Con este
valor se vuelve a plantear una última regresión lineal, esta vez tomando 
$x_i = \log \left(d_i-\varepsilon\right)$ e $y_i = \log C_i$. Nuevamente se
busca que el valor de la pendiente sea cercano a \num{2.5}. Los valores de los
cálculos se encuentran en la tabla \ref{tab:4:regresion-final}:

\begin{table}[H]
    \centering
    \csvreader[
        table head = \toprule $B$ & $A$ \\\midrule,
        table foot = \bottomrule,
        tabular = cc
    ]{data/regresion2.csv}{}{
        \num{\csvcolviii \pm \csvcolx} & \num{\csvcolvii \pm \csvcolix}
    }
    \caption{Regresión lineal para el modelo ajustado, aplicando $\log$}
    \label{tab:4:regresion-final}
\end{table}

Según la tabla \ref{tab:4:regresion-final} el valor de la pendiente es
ahora consistente con el valor esperado, por lo que \textbf{sí se verifica
la funcionalidad predicha por el análisis dimensional}. Ahora, sabiendo que 
$A = \log \left( k\,\delta\,g^{1/2} \right)$ (por la ec. \ref{ec:3:caudal-log}),
se encuentra el valor de la constante
$k = \csvreader[]{data/regresion2.csv}{}{\num{\csvcolxi \pm \csvcolxii}}$
(adimensional), mediante las ecuaciones \ref{ec:4:k} y
\ref{ec:4:dk} (despreciando la incertidumbre en la constante $g$).

\begin{equation}
    \label{ec:4:k}
    k = 10^{A}\, \delta^{-1}\, g^{-1/2}
\end{equation}

\begin{equation}
    \label{ec:4:dk}
    \Delta k =
    \left| \frac{2}{5} B^{3/2}\, \delta^{-1}\, g^{-1/2} \right| \cdot \Delta B
    + \left| -B^{5/2}\, \delta^{-2}\, g^{-1/2} \right| \cdot \Delta \delta
\end{equation}

Se puede concluír que el modelo planteado por la ecuación \ref{ec:3:caudal} es
válido. La gráfica de la figura \ref{fig:4:funcion} ilustra esta función con 
los datos calculados anteriormente.

\begin{figure}[H]
    \centering
    \input{img/funcion.tikz}
    \caption{Representación del modelo matemático y los datos experimentales.}
    \label{fig:4:funcion}
\end{figure}

}
