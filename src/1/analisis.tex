\subsection{Análisis de datos}

El gráfico de la fig. \ref{fig:1-analisis:ganancia} muestra una comparación
entre el valor de ganancia teórico (calculado con las ecuaciones 
\ref{ec:1-teoria:ganancia} y \ref{ec:1-teoria:err-ganancia}) y aquellos 
medidos en la práctica para los distintos valores de resistencia. Como puede
observarse, todos los valores de ganancia medidos caen cómodamente dentro
del margen de error previsto. Esto sugiere que las ecuaciones de ganancia
que se utilizaron para la configuración inversora del amplificador operacional
permiten obtener una buena aproximación al funcionamiento real del circuito.

\begin{figure}[H]
    \centering
    \input{img/1/ganancia.tikz}
    \caption{Valores de ganancia medidos vs. teóricos}
    \label{fig:1-analisis:ganancia}
\end{figure}

Dado que los valores medidos de tensión en la terminal no inversora son
mayores que $0$ en dos de los tres casos estudiados\footnote{Es altamente
probable que $v_+$ estuviera en el orden de los \si{\micro\volt}, en cuyo
caso nos fue imposible medir con el multímetro del que disponíamos.}, se
comprueba la explicación ofrecida en la sección \ref{sec:1-teoria} sobre la
utilización del resistor $R_3$. Si la resistencia de entrada del operacional
fuera infinita, esta tensión sería 0.
