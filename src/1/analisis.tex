\subsection{Análisis de datos}

El gráfico de la fig. \ref{fig:1-analisis:ganancia} muestra una comparación
entre el valor de ganancia teórico (calculado con las ecuaciones 
\ref{ec:1-teoria:ganancia} y \ref{ec:1-teoria:err-ganancia}) y aquellos 
medidos en la práctica para los distintos valores de resistencia. Como puede
observarse, dos de los tres valores de ganancia son consistentes con el valor
predicho por la teoría. Esto sugiere que las ecuaciones de ganancia
que se utilizaron para la configuración inversora del amplificador operacional
permiten obtener una buena aproximación al funcionamiento real del circuito.
La ganancia calculada para $R_L = \SI{1}{\kilo\ohm}$ difiere mucho
de los demás valores, cuando sería esperable que todos los valores estuvieran
relativamente cerca entre sí ya que la ganancia del circuito no debería
depender del valor de $R_L$. De todas formas, promediando los datos de la tabla
\ref{tab:1-teoria:ganancia-experimental} y calculando la incertidumbre como
$\sqrt{3}\,\sigma_{n-1}$ se obtiene una ganancia
$A = \num{-1.3 \pm 0.3}$, que aún con un error de $23\%$ resulta ser 
totalmente consistente con las estimaciones teóricas.

\begin{figure}[H]
    \centering
    \input{img/1/ganancia.tikz}
    \caption{Valores de ganancia medidos vs. teóricos}
    \label{fig:1-analisis:ganancia}
\end{figure}

Dado que los valores medidos de tensión en la terminal no inversora son
mayores que $0$ en dos de los tres casos estudiados, se comprueba la
explicación ofrecida en la sección \ref{sec:1-teoria} sobre la
utilización del resistor $R_3$. Si la resistencia de entrada del operacional
fuera infinita, esta tensión sería 0.
