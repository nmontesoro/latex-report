\subsection{Análisis de datos}

El gráfico de la fig. \ref{fig:1-analisis:ganancia} muestra una comparación
entre el valor de ganancia teórico (calculado con las ecuaciones 
\ref{ec:1-teoria:ganancia} y \ref{ec:1-teoria:err-ganancia}) y aquellos 
medidos en la práctica para los distintos valores de resistencia. Como puede
observarse, todos los valores de ganancia medidos caen cómodamente dentro
del margen de error previsto. Esto sugiere que las ecuaciones de ganancia
que se utilizaron para la configuración inversora del amplificador operacional
permiten obtener una buena aproximación al funcionamiento real del circuito.

\begin{figure}[H]
    \centering
    \input{img/1/ganancia.tikz}
    \caption{Valores de ganancia medidos vs. teóricos}
    \label{fig:1-analisis:ganancia}
\end{figure}

