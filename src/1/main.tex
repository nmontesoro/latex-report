\section{Primera parte: amplificador inversor}

Se arma el circuito de la fig. \ref{fig:1:esquema} en un \textit{protoboard}
utilizando resistencias $R_1 = \SI{10}{\kilo\ohm}$, $R_2 = \SI{15}{\kilo\ohm}$ y
$R_3 = \SI{10}{\kilo\ohm}$. Luego se miden las tensiones $V_i$, $V_o$, $V_{+}$,
$V_{-}$ y $V_{+} - V_{-}$, y las corrientes que atraviesan a las resistencias
para valores de $R_L$ de 1, 10 y \SI{47}{\kilo\ohm}. 

\subsection{Resolución teórica}

Considerando al operacional como un elemento ideal (ver sección \ref{sec:intro})
se sabe que su resistencia de entrada es infinita. Por tal motivo no circulará
corriente a través de la resistencia $R_3$, lo cual implica que no habrá caída
de tensión alguna entre sus terminales. Efectivamente es como tener el circuito
descrito en la sección \ref{sec:intro:opamp-inversor} y por ende se puede usar
la ecuación \ref{ec:intro:opamp-inversor}. Entonces queda:

\begin{equation}
    \label{ec:1-teoria:ganancia}
    A = -\frac{R_2}{R_1}
\end{equation}

\begin{equation}
    \label{ec:1-teoria:err-ganancia}
    \Delta A = \left| - \frac{1}{R_1} \right| \Delta R_2
             + \left| - \frac{R_2}{{R_1}^2} \right| \Delta R_1
\end{equation}

Se observa que el valor de la ganancia \textbf{no depende del valor de la
resistencia de carga}. Considerando que $\Delta R_i = 0.2\,R_i\ \forall\ i$, ya
que los resistores utilizados tienen una tolerancia de $\pm 20\%$, el análisis
anterior resulta en:

\[
    \input{src/1/snippets/ganancia-analitica}
\]

Como puede observarse, el valor de la ganancia no debería verse afectado por el
resistor $R_3$. Esto se debe a la alta resistencia de entrada del operacional
(que de hecho es infinita en este modelo ideal).

Si bien el modelo ideal no admite que circule corriente entre masa y la
terminal no inversora, debido a esta resistencia infinita, en la realidad
el operacional tiene una ganancia \textit{finita}, pero muy grande (en torno a
los $10^{12}$ \si{\ohm} en el caso del TL072 \cite[pág. 5]{datasheet-tl072}).
Por este motivo se produce una corriente mínima, llamada corriente de
polarización (\textit{input bias current}), que depende del dispositivo que se
utilice y que provoca una caída de potencial entre los terminales de $R_3$.
Para el TL072, esta corriente está entre 65 y \SI{200}{\pico\ampere} 
\cite[pág. 5]{datasheet-tl072}. Es por esta razón que se aconseja no
conectar la terminal no inversora directamente a masa \cite[pág. 252]{AOE}.


\subsection{Datos obtenidos}

Se midieron los siguientes valores de resistencias:

\begin{itemize}
        \input{src/1/snippets/resistencias}
\end{itemize}

Los valores medidos de $R_L$, comparados con sus valores nominales, se encuentran en la tabla \ref{tab:1-datos:resistencias-l}.

También se tomaron mediciones de las distintas tensiones presentes en el circuito (tablas \ref{tab:1-datos:tensiones-1} y \ref{tab:1-datos:tensiones-2}) que fueron utilizadas para calcular las corrientes que circulan por cada resistor y cuyos valores se encuentran en la tabla \ref{tab:1-datos:corrientes}.

\begin{table}[H]
    \centering
    \begin{tabular}{@{}rr@{}}
        \toprule
        $R_L$ (nominal, \si{\kilo\ohm}) & $R_L$ (medido, \si{\kilo\ohm})  \\
        \midrule
        \input{src/1/tables/resistencias-l}
    \end{tabular}
    \caption{Valores medidos de $R_L$}
    \label{tab:1-datos:resistencias-l}
\end{table}

\begin{table}[H]
    \centering
    \begin{tabular}{@{}rrrrrr@{}}
        \toprule
        $R_L$ (\si{\kilo\ohm}) & $v_i$ (\si{\volt}) & $v_o$ (\si{\volt}) & 
            $v_+$ (\si{\milli\volt}) & $v_-$ (\si{\milli\volt}) &
            $v_+ - v_-$ (\si{\milli\volt}) \\
        \midrule
        \input{src/1/tables/tensiones-1}
    \end{tabular}
    \caption{Tensiones medidas en el circuito (parte 1)}
    \label{tab:1-datos:tensiones-1}
\end{table}

\begin{table}[H]
    \centering
    \begin{tabular}{@{}rrrrrr@{}}
        \toprule
        $R_L$ (\si{\kilo\ohm}) & $V_{R_1}$ (\si{\volt}) & $V_{R_2}$ (\si{\volt})& $V_{R_3}$ (\si{\milli\volt}) & $V_{R_L}$ (\si{\volt}) \\
        \midrule
        \input{src/1/tables/tensiones-2}
    \end{tabular}
    \caption{Tensiones medidas en el circuito (parte 2)}
    \label{tab:1-datos:tensiones-2}
\end{table}

\begin{table}[H]
    \centering
    \begin{tabular}{@{}rrrrr@{}}
        \toprule
        $R_L$ (\si{\kilo\ohm}) & $I_{R_1}$ (\si{\milli\ampere}) & $I_{R_2}$ (\si{\milli\ampere}) & $I_{R_3}$ (\si{\milli\ampere}) & $I_{R_L}$ (\si{\milli\ampere}) \\
        \midrule
        \input{src/1/tables/corrientes}
    \end{tabular}
    \caption{Corrientes calculadas en el circuito}
    \label{tab:1-datos:corrientes}
\end{table}

La ganancia del operacional puede calcularse experimentalmente como

\begin{equation}
    \label{ec:1-teoria:ganancia-experimental}
    A = \frac{v_o}{v_i}
\end{equation}

\begin{equation}
    \label{ec:1-teoria:err-ganancia-experimental}
    \Delta A = \left| \frac{1}{v_i} \right| \Delta v_o + \left| -\frac{v_o}{{v_i}^2} \right| \Delta v_i
\end{equation}

Los resultados de esta operación se encuentran en la tabla \ref{tab:1-teoria:ganancia-experimental}.


\begin{table}[H]
    \centering
    \begin{tabular}{@{}rr@{}}
        \toprule
        $R_L$ (\si{\kilo\ohm}) & $A$ \\
        \midrule
        \input{src/1/tables/ganancia-experimental}
    \end{tabular}
    \caption{Valores experimentales de ganancia}
    \label{tab:1-teoria:ganancia-experimental}
\end{table}


% \subsection{Análisis de datos}

La fig. \ref{fig:1-analisis:ganancia} compara la ganancia teórica con los valores obtenidos en la práctica.

\begin{figure}[H]
    \includegraphics[width=\textwidth]{img/1-analisis-ganancia.eps}
    \caption{Ganancia teórica vs. ganancia experimental}
    \label{fig:1-analisis:ganancia}
\end{figure}

