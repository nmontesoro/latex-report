\subsection{Resolución teórica}
\label{sec:1-teoria}

Considerando al operacional como un elemento ideal (ver sección \ref{sec:intro})
se sabe que su resistencia de entrada es infinita. Por tal motivo no circula
corriente a través de la resistencia $R_3$, lo cual implica que no hay caída
de tensión alguna entre sus terminales. Efectivamente es como tener el circuito
descrito en la sección \ref{sec:intro:opamp-inversor} y por ende se puede usar
la ecuación \ref{ec:intro:opamp-inversor}. Entonces queda:

\begin{equation}
    \label{ec:1-teoria:ganancia}
    A = -\frac{R_2}{R_1}
\end{equation}

\begin{equation}
    \label{ec:1-teoria:err-ganancia}
    \Delta A = \left| - \frac{1}{R_1} \right| \Delta R_2
             + \left| - \frac{R_2}{{R_1}^2} \right| \Delta R_1
\end{equation}

Se observa que el valor de la ganancia \textbf{no depende del valor de la
resistencia de carga}. Considerando que $\Delta R_i = 0.05\,R_i\ \forall\ i$, ya
que los resistores utilizados tienen una tolerancia de $\pm 5\%$, el análisis
anterior resulta en:

\[
    \input{src/1/snippets/ganancia-analitica}
\]

Como puede observarse, el valor de la ganancia no debe verse afectado por el
resistor $R_3$. Esto se debe a la alta resistencia de entrada del operacional
(que de hecho es infinita en este modelo ideal).

Si bien el modelo ideal no admite que circule corriente entre masa y la
terminal no inversora, debido a esta resistencia infinita, en la realidad
el operacional tiene una ganancia \textit{finita}, pero muy grande (en torno a
los 10 \si{\tera\ohm} en el caso del TL072 \cite[pág. 5]{datasheet-tl072}).
Por este motivo se produce una corriente mínima, llamada corriente de
polarización (\textit{input bias current}), que depende del dispositivo que se
utilice y que provoca una caída de potencial entre los terminales de $R_3$.
Para el TL072, esta corriente está entre 65 y \SI{200}{\pico\ampere} 
\cite[pág. 5]{datasheet-tl072}. Es por esta razón que se aconseja no
conectar la terminal no inversora directamente a masa \cite[pág. 252]{AOE}.
