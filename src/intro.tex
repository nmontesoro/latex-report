\section{Introducción}
\label{sec:intro}

El \textbf{amplificador operacional} es un dispositivo que se comporta como una fuente de tensión controlada por tensión, con un valor de ganancia alto \cite[pág. 107]{huelsman}. También puede servir para generar una corriente controlada (por tensión o por corriente), y es capaz de sumar, amplificar, integrar y diferenciar señales de entrada \cite[pág. 176]{FDCE}. 

% TODO: Agregar figura del opamp, explicar sus terminales

Un \textbf{amplificador operacional ideal} tiene las siguientes características:

\begin{itemize}
    \item Ganancia infinita, $A\approx\infty$
    \item Resistencia de entrada infinita, $R_{\text{in}} \approx \infty$
    \item Resistencia de salida cero, $R_{\text{out}} \approx 0$
\end{itemize}

Si bien al utilizar estas suposiciones se consigue una respuesta aproximada, la mayoría de los operacionales tienen ganancias e impedancias de entrada tales que este análisis se torna aceptable \cite[pág. 180]{FDCE}.

Un amplificador operacional se puede conectar de diversas maneras:

\begin{itemize}
    \item Amplificador inversor
    \item Amplificador no inversor
    \item Amplificador seguidor de tensión
    \item Amplificador sumador
    \item Amplificador diferencial
\end{itemize}

Además es posible conectar varios amplificadores en cascada, en un arreglo de dos o más circuitos dispuestos uno tras otro, de manera que la salida de uno conforme la entrada del siguiente \cite[pág. 191]{FDCE}.

\subsection{Amplificador inversor}
\label{sec:intro:opamp-inversor}

\begin{wrapfigure}{r}{0.4\textwidth}
    \centering
    \begin{circuitikz}
    \node[op amp] at (0, 0) (opamp) {};

    \draw(opamp.-) -- ++(-0.7, 0) node[label={above:$v_-$}]{}
    to[R=$R_1$] ++(-1.5, 0)
    to[short, -o] ++(-0.2, 0) node[left]{$v_i$};

    \draw(opamp.+) -- ++(-0.2, 0) node[label={left:$v_+$}]{}
    to[short] ++(0, -0.5)
    node[ground]{};

    \draw(opamp.-) -- ++(-0.2, 0)
    to[short, *-] ++(0, 1.5)
    to[R=$R_2$] ++(2.8, 0) coordinate(co)
    to[short] (opamp.out -| co) coordinate(co1) -- (opamp.out);

    \draw(co1) 
    to[short, *-o] ++(0.4, 0) node[right]{$v_o$};
\end{circuitikz}

    \caption{Amplificador inversor}
    \label{fig:intro:opamp-inversor}
\end{wrapfigure}


Para configurar al amplificador operacional como inversor, se conecta a tierra la entrada no inversora. La señal de entrada $v_i$ se conecta a la entrada inversora a través de $R_1$, y se conecta un resistor $R_2$ entre la entrada inversora y la salida (éste resistor es ocasionalmente llamado \textit{resistor de retroalimentación}) \cite[pág. 181]{FDCE}.
Si asumimos que se trata de un operacional ideal entonces $v_{-} = v_{+}$, y $v_{+} = \SI{0}{\volt}$ por estar conectada a masa la terminal no inversora. Siendo $v_o$ la tensión de salida y $A_v$ la ganancia en tensión, al aplicar la LCK en la entrada inversora se llega a que

\begin{equation}
    \label{ec:intro:opamp-inversor}
    v_o = - \frac{R_2}{R_1} \, v_i = A_v v_i
\end{equation}

Por lo tanto se demuestra que \textbf{un amplificador inversor invierte la polaridad de entrada a la vez que amplifica la señal}.

\subsection{Amplificador no inversor}
\label{sec:intro:opamp-noinversor}

% TODO: Agregar figuras

\begin{wrapfigure}{r}{0.4\textwidth}
    \centering
    \begin{circuitikz}
    \node[op amp] at (0, 0) (opamp) {};

    \draw(opamp.-) -- ++(-0.5, 0) node[label={above:$v_-$}]{}
    to[R=$R_1$] ++(-2, 0)
    to[short] ++(0, -1.5) node[ground]{};

    \draw(opamp.+) -- ++(-0.2, 0) node[label={left:$v_+$}]{}
    to[short, -o] ++(0, -1)
    node[right]{$v_i$};

    \draw(opamp.-) -- ++(-0.2, 0)
    to[short, *-] ++(0, 1.5)
    to[R=$R_2$] ++(2.8, 0) coordinate(co)
    to[short] (opamp.out -| co) coordinate(co1) -- (opamp.out);

    \draw(co1) 
    to[short, *-o] ++(0.4, 0) node[right]{$v_o$};
\end{circuitikz}

    \caption{Amplificador no inversor}
    \label{fig:intro:opamp-no-inversor}
\end{wrapfigure}

En este caso se aplica la tensión de entrada a la terminal no inversora, conectando $R_1$ entre masa y la terminal inversora. Dado que $v_{-} = v_{+} = v_i$, al aplicar la LCK en la terminal inversora se encuentra la relación entre la tensión de salida y la tensión de entrada, que es:

\begin{equation}
    \label{ec:intro:opamp-noinversor}
    v_o = \left(1 + \frac{R_2}{R_1}\right) \, v_i = A_v v_i
\end{equation}

Como la ganancia es un número positivo, \textbf{un amplificador no inversor produce una señal de salida con la misma polaridad que la señal de entrada}.

A su vez, si se diera el caso en que $R_2 = \SI{0}{\ohm}$ (cortocircuito) o $R_1 = \infty$ (circuito abierto), o ambos, se configura el circuito conocido como \textbf{seguidor de tensión} o \textbf{amplificador de ganancia unitaria}, dado que $v_o = v_i$. Tiene una impedancia de entrada muy alta, por lo que puede ser utilizado para aislar un circuito de otro \cite[pág. 184]{FDCE}.
