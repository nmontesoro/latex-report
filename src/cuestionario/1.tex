\question{Construya un histograma con sus medidas. Comente acerca de la 
forma obtenida}

\answer{
    Se calcula la cantidad de divisiones según la regla de Sturges
    \cite{sturges}. Siendo $M = 200$ el tamaño de la muestra,

    \begin{equation}
        \label{ec:cuestionario:sturges}
        c = 1 + \log_2 M = 9
    \end{equation}

    Luego se construye el histograma de la fig.
    \ref{fig:cuestionario:histograma} usando Octave. Se puede observar una
    forma de distribución Gaussiana de la variable $T$.

    \begin{figure}[H]
        \centering
        \input{img/histograma.tikz}
        \caption{Histograma}
        \label{fig:cuestionario:histograma}
    \end{figure}
}
