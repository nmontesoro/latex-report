\question{¿Cómo continuarían los gráficos de los incisos 2 y 3 a medida que la
cantidad de mediciones aumenta hasta hacerse infinita?}

\answer{
    Por un lado, si hiciéramos una cantidad infinita de mediciones, el gráfico
    del inciso 2 seguiría tendiendo al valor de media y su error, es decir, la
    desviación de la media, tendería a 0.

    Por otro lado, con una cantidad infinita de mediciones, el gráfico del
    inciso 3 seguiría tendiendo al mismo valor que graficamos, y el error
    también sería aproximadamente igual. De hecho, con 200 mediciones ya es
    posible observar que hacia el final de la curva el error tiende a un valor
    constante y a partir de ese punto no mejora demasiado, por lo que no
    tendría sentido hacer muchas mediciones más.
}
