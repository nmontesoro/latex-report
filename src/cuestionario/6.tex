\question{Calcule e informe la constante de la gravedad para la varilla que 
    midió doscientas veces con su error.}

\answer{
    Despejando $g$ de la ecuación \ref{ec:intro:periodo} y propagando el error
    nos queda:

    \begin{equation}
        \label{ec:cuestionario:gravedad}
        g = \frac{8 \pi^2 L}{3 T^2}
    \end{equation}

    \begin{equation}
        \label{ec:cuestionario:error-gravedad}
        \Delta g = \frac{8}{3} \pi^2 \left(
                   \frac{\Delta L}{T^2} + \frac{2L}{T^3}
                   \Delta T \right)
    \end{equation}

    La longitud de la primera varilla es
    $L = \input{src/cuestionario/6/longitud}$. Tomando 
    $T = \bar{T} + \frac{3\,\sigma_{n-1}}{\sqrt{N}} = \input{src/cuestionario/6/periodo}$ se
    obtiene

    \[
        g = \input{src/cuestionario/6/gravedad}
    \]
}
