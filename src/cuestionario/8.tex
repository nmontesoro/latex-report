\question{
    Suponiendo que el cronómetro que usted utilizó demora 12 centésimas de
    segundo en detenerse al realizar la medición, esquematice para este caso
    la distribución de las mediciones y marque en el mismo gráfico la media de
    la distribución y el verdadero valor del período de oscilación.
}

\answer{
    En este caso estaríamos introduciendo un error sistemático de 12 centésimas
    a todas las mediciones. Por lo tanto, la media quedaría desplazada 12
    centésimas hacia la derecha respecto de la distribución original, de forma
    similar a lo que se observa en la fig. \ref{fig:cuestionario:gauss}. En
    cuanto a la forma del gráfico, seguiría siendo una campana gaussiana ya que
    las desviaciones entre las mediciones no variarán con el error introducido.

    \begin{figure}[H]
        \centering
        \input{img/gauss.tikz}
        \caption{Distribución gaussiana}
        \label{fig:cuestionario:gauss}
    \end{figure}
}
