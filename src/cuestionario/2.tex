\question{Construya un gráfico de la media muestral en función de la 
cantidad de medidas, en el orden que estas fueron tomadas. Incluya
barras de error en los puntos.}

\answer{
    La media muestral se calcula según la ecuación \ref{ec:cuestionario:media},
    mientras que el error de la media (su desviación) se calcula como la
    ecuación \ref{ec:cuestionario:error-media},

    \begin{equation}
        \label{ec:cuestionario:media}
        \bar{X} = \frac{\sum x_i}{N}
    \end{equation}

    \begin{equation}
        \label{ec:cuestionario:error-media}
        \sigma_{\bar{X}} = \frac{3 \, \sigma_{n-1}}{\sqrt{N}}
    \end{equation}
    donde $\sigma_{n-1}$ es la desviación estándar de la muestra (ver
    siguiente ejercicio) y $N$ es el tamaño de la muestra.

    Usando las dos ecuaciones anteriores se consigue el gráfico de la
    fig. \ref{fig:cuestionario:media}.

    \begin{figure}[H]
        \centering
        \input{img/media.tikz}
        \caption{Media muestral}
        \label{fig:cuestionario:media}
    \end{figure}
}
