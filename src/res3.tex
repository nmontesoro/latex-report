\question{
Diga si se verifica o no la funcionalidad predicha por el análisis dimensional.
}

\answer{

Según el análisis dimensional del ejercicio 1, el caudal debería responder a
la ecuación \ref{ec:3:caudal}, donde $C$ es el caudal másico, $K$ es una 
constante, $g$ es la constante de la gravedad y $d$ el diámetro del orificio.
Si se aplica $\log$ a ambos lados y se utilizan las propiedades adecuadas, se 
obtiene la ecuación \ref{ec:3:caudal-log}:

\begin{equation}
    \label{ec:3:caudal}
    C = K\,\delta \,g^{1/2} \,d^{5/2}
\end{equation}

\begin{equation}
    \label{ec:3:caudal-log}
    \log C = \frac{5}{2} \log d + \log \left( K\delta g^{1/2} \right)
\end{equation}

Graficando $10^{\log C}$ en escala logarítmica se obtendría una recta de 
pendiente $\frac{5}{2}$, lo que indica que si se aplica regresión lineal a los
datos de la fig. \ref{fig:2:caudales} debería obtenerse una pendiente $B$
cercana a \num{2.5} para verificar la funcionalidad del análisis dimensional.
Siendo $N = 4$ y $x_i$ e $y_i$ los valores de diámetro y caudal de las tablas
\ref{tab:datos:diametros} y \ref{tab:2:caudales} luego de aplicarles 
$\log_{10}$, se obtiene:

\newcommand{\denominador}{N \sum {x_i}^2 - \left( \sum x_i \right)^2}
\newcommand{\columnacsv}[1]{
    \csvreader[]{data/regresion.csv}{}{#1}
}

\begin{equation}
    \label{ec:3:reg-B}
    B = \frac{N\, \sum x_i y_i - \sum x_i \sum y_i}{\denominador} 
    = \columnacsv{\num{\csvcoliii}}
\end{equation}

\vspace{5mm}

\begin{equation}
    \label{ec:3:reg-A}
    A = \frac{\sum{x_i}^2 \sum{y_i} - \sum x_i \sum x_i y_i}{\denominador}
    = \columnacsv{\num{\csvcoliv}}
\end{equation}

\vspace{5mm}

\begin{equation}
    \label{ec:3:reg-oY}
    \sigma_y = \sqrt{ \frac{1}{N-2} \sum \left ( y_i - A - Bx_i \right) ^2 }
    = \columnacsv{\num{\csvcolv}}
\end{equation}

\vspace{5mm}

\begin{equation}
    \label{ec:3:reg-oB}
    \sigma_B = \sigma_y \sqrt{ \frac{N}{\denominador} }
    = \columnacsv{\num{\csvcolvi}}
\end{equation}

\vspace{5mm}

\begin{equation}
    \label{ec:3:reg-oA}
    \sigma_A = \sigma_y \sqrt{ \frac{\sum {x_i} ^ 2}{\denominador} }
    = \columnacsv{\num{\csvcolvii}}
\end{equation}

\vspace{5mm}

Entonces el valor de pendiente por regresión lineal es

\begin{equation*}
    B + 3\, \frac{\sigma_B}{\sqrt{N}}
    = \columnacsv{\num{\csvcoliii \pm \csvcolviii}}
\end{equation*}
que es inconsistente con el valor buscado de \num{2.5}. Por lo tanto 
\textbf{no es posible verificar la funcionalidad del análisis dimensional.}
La recta encontrada está graficada en la fig. \ref{fig:3:regresion}.

\begin{figure}[H]
    \centering
    \input{img/regresion.tikz}
    \caption{Regresión lineal sobre los datos obtenidos.}
    \label{fig:3:regresion}
\end{figure}

}

