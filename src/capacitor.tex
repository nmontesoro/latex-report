\subsection{Mediciones del capacitor cilíndrico}

Se obtuvieron los datos de la tabla \ref{tab:capacitor:datos}, con los que se 
puede hallar la recta de mejor ajuste utilizando el mismo método de la sección
anterior. En este caso, $f^{-1} = mL + b$, siendo
\csvreader[no head]{data/calculos-2.csv}{}{
    $m = \SI{\csvcolvi \pm \csvcolviii}{\per\cm\kilo\hertz}$ y
    $b = \SI{\csvcolvii \pm \csvcolix}{\per\kilo\hertz}$.
} Tanto los datos como la recta de mejor ajuste se encuentran en el gráfico de la
fig. \ref{fig:capacitor:rectas}.

Los diámetros exterior e interior de los cilindros fueron 
\csvreader[no head]{data/calculos-3.csv}{}{
    $d_e = \SI{\csvcolv \pm \csvcolvi}{\mm}$ y
    $d_i = \SI{\csvcoli \pm \csvcolii}{\mm}$} respectivamente,
con lo que los radios son 
\csvreader[no head]{data/calculos-3.csv}{}{
    $r_e = \SI{\csvcolvii \pm \csvcolviii}{\mm}$ y
    $r_i = \SI{\csvcoliii \pm \csvcoliv}{\mm}$}.
\vspace{10mm}
\begin{table}[H]
    \centering
    \csvreader[
        no head,
        tabular = rr,
        table head = \toprule $L$ (\si{\cm}) & $f$ (\si{\kilo\hertz}) \\\midrule,
        table foot = \bottomrule,
    ]{data/longitudes.csv}{}{
        \num{\csvcoli \pm \csvcolii} & \num{\csvcoliii \pm \csvcoliv}
    }
    \caption{Datos obtenidos}
    \label{tab:capacitor:datos}
\end{table}

\begin{figure}[H]
    \centering
    \subsection{Mediciones del capacitor cilíndrico}

Se obtuvieron los datos de la tabla \ref{tab:capacitor:datos}, con los que se 
puede hallar la recta de mejor ajuste utilizando el mismo método de la sección
anterior. En este caso, $f^{-1} = mL + b$, siendo
\csvreader[no head]{data/calculos-2.csv}{}{
    $m = \SI{\csvcolvi \pm \csvcolviii}{\per\cm\kilo\hertz}$ y
    $b = \SI{\csvcolvii \pm \csvcolix}{\per\kilo\hertz}$.
} Tanto los datos como la recta de mejor ajuste se encuentran en el gráfico de la
fig. \ref{fig:capacitor:rectas}.

Los diámetros exterior e interior de los cilindros fueron 
\csvreader[no head]{data/calculos-3.csv}{}{
    $d_e = \SI{\csvcolv \pm \csvcolvi}{\mm}$ y
    $d_i = \SI{\csvcoli \pm \csvcolii}{\mm}$} respectivamente,
con lo que los radios son 
\csvreader[no head]{data/calculos-3.csv}{}{
    $r_e = \SI{\csvcolvii \pm \csvcolviii}{\mm}$ y
    $r_i = \SI{\csvcoliii \pm \csvcoliv}{\mm}$}.
\vspace{10mm}
\begin{table}[H]
    \centering
    \csvreader[
        no head,
        tabular = rr,
        table head = \toprule $L$ (\si{\cm}) & $f$ (\si{\kilo\hertz}) \\\midrule,
        table foot = \bottomrule,
    ]{data/longitudes.csv}{}{
        \num{\csvcoli \pm \csvcolii} & \num{\csvcoliii \pm \csvcoliv}
    }
    \caption{Datos obtenidos}
    \label{tab:capacitor:datos}
\end{table}

\begin{figure}[H]
    \centering
    \subsection{Mediciones del capacitor cilíndrico}

Se obtuvieron los datos de la tabla \ref{tab:capacitor:datos}, con los que se 
puede hallar la recta de mejor ajuste utilizando el mismo método de la sección
anterior. En este caso, $f^{-1} = mL + b$, siendo
\csvreader[no head]{data/calculos-2.csv}{}{
    $m = \SI{\csvcolvi \pm \csvcolviii}{\per\cm\kilo\hertz}$ y
    $b = \SI{\csvcolvii \pm \csvcolix}{\per\kilo\hertz}$.
} Tanto los datos como la recta de mejor ajuste se encuentran en el gráfico de la
fig. \ref{fig:capacitor:rectas}.

Los diámetros exterior e interior de los cilindros fueron 
\csvreader[no head]{data/calculos-3.csv}{}{
    $d_e = \SI{\csvcolv \pm \csvcolvi}{\mm}$ y
    $d_i = \SI{\csvcoli \pm \csvcolii}{\mm}$} respectivamente,
con lo que los radios son 
\csvreader[no head]{data/calculos-3.csv}{}{
    $r_e = \SI{\csvcolvii \pm \csvcolviii}{\mm}$ y
    $r_i = \SI{\csvcoliii \pm \csvcoliv}{\mm}$}.
\vspace{10mm}
\begin{table}[H]
    \centering
    \csvreader[
        no head,
        tabular = rr,
        table head = \toprule $L$ (\si{\cm}) & $f$ (\si{\kilo\hertz}) \\\midrule,
        table foot = \bottomrule,
    ]{data/longitudes.csv}{}{
        \num{\csvcoli \pm \csvcolii} & \num{\csvcoliii \pm \csvcoliv}
    }
    \caption{Datos obtenidos}
    \label{tab:capacitor:datos}
\end{table}

\begin{figure}[H]
    \centering
    \subsection{Mediciones del capacitor cilíndrico}

Se obtuvieron los datos de la tabla \ref{tab:capacitor:datos}, con los que se 
puede hallar la recta de mejor ajuste utilizando el mismo método de la sección
anterior. En este caso, $f^{-1} = mL + b$, siendo
\csvreader[no head]{data/calculos-2.csv}{}{
    $m = \SI{\csvcolvi \pm \csvcolviii}{\per\cm\kilo\hertz}$ y
    $b = \SI{\csvcolvii \pm \csvcolix}{\per\kilo\hertz}$.
} Tanto los datos como la recta de mejor ajuste se encuentran en el gráfico de la
fig. \ref{fig:capacitor:rectas}.

Los diámetros exterior e interior de los cilindros fueron 
\csvreader[no head]{data/calculos-3.csv}{}{
    $d_e = \SI{\csvcolv \pm \csvcolvi}{\mm}$ y
    $d_i = \SI{\csvcoli \pm \csvcolii}{\mm}$} respectivamente,
con lo que los radios son 
\csvreader[no head]{data/calculos-3.csv}{}{
    $r_e = \SI{\csvcolvii \pm \csvcolviii}{\mm}$ y
    $r_i = \SI{\csvcoliii \pm \csvcoliv}{\mm}$}.
\vspace{10mm}
\begin{table}[H]
    \centering
    \csvreader[
        no head,
        tabular = rr,
        table head = \toprule $L$ (\si{\cm}) & $f$ (\si{\kilo\hertz}) \\\midrule,
        table foot = \bottomrule,
    ]{data/longitudes.csv}{}{
        \num{\csvcoli \pm \csvcolii} & \num{\csvcoliii \pm \csvcoliv}
    }
    \caption{Datos obtenidos}
    \label{tab:capacitor:datos}
\end{table}

\begin{figure}[H]
    \centering
    \input{img/capacitor.tikz}
    \caption{Datos y recta de mejor ajuste}
    \label{fig:capacitor:rectas}
\end{figure}

    \caption{Datos y recta de mejor ajuste}
    \label{fig:capacitor:rectas}
\end{figure}

    \caption{Datos y recta de mejor ajuste}
    \label{fig:capacitor:rectas}
\end{figure}

    \caption{Datos y recta de mejor ajuste}
    \label{fig:capacitor:rectas}
\end{figure}
