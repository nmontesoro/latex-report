\subsection{Cálculo de $\varepsilon$}

A partir de la recta de ajuste de calibración ($f^{-1}$ vs. $C$) y la recta de
ajuste de la sección anterior ($f^{-1}$ vs. $L$) se obtienen las ecuaciones
\ref{ec:epsilon:c} y \ref{ec:epsilon:error-c}, que relacionan la capacidad $C$
del capacitor cilíndrico con su longitud $L$, los valores de $m$ y $b$ del
inciso anterior, y los parámetros $\alpha$ y $C_o$ de la sección
\ref{sec:calibracion}.

\begin{equation}
    \label{ec:epsilon:c}
    C = ML + B,\quad\quad
    M = \alpha m, \quad\quad
    B = \alpha b - C_o
\end{equation}

\begin{equation}
    \label{ec:epsilon:error-c}
    \Delta C = \left| M \right| \Delta L 
             + \left| L \right| \Delta M
             + \Delta B, \quad\quad
\end{equation}

\vspace{10mm}
Con lo anterior se puede generar la gráfica de la fig. 
\ref{fig:epsilon:cvsl}.

\begin{figure}[H]
    \centering
    \input{img/cvsl.tikz}
    \caption{Capacidad del capacitor cilíndrico según su longitud}
    \label{fig:epsilon:cvsl}
\end{figure}

\vspace{10mm}
Según la teoría esta última gráfica debería responder a la ecuación 
\ref{ec:epsilon:teoria}:

\begin{equation}
    \label{ec:epsilon:teoria}
    C = \frac{2\pi\varepsilon}{\ln\left(\frac{r_e}{r_i}\right)} L
\end{equation}

\vspace{5mm}
Con lo cual deberían cumplirse las siguientes condiciones en nuestro modelo
lineal:

\[
    M = \frac{2\pi\varepsilon}{\ln\left(\frac{r_e}{r_i}\right)}, 
    \quad\quad B = 0
\]

\vspace{5mm}
Como se cuenta con el valor de $M$ se puede despejar $\varepsilon$, quedando

\begin{equation}
    \label{ec:epsilon:epsilon}
    \varepsilon = \frac{M}{2\pi} \ln \left(\frac{r_e}{r_i}\right)
\end{equation}
\vspace{5mm}
\begin{equation}
    \label{ec:epsilon:error-epsilon}
    \Delta \varepsilon = \frac{1}{2\pi} \left[ 
            \left| \ln\left(\frac{r_e}{r_i}\right) \right| \Delta M +
            \left| \frac{M}{r_e} \right| \Delta r_i + 
            \left| \frac{-M}{r_i} \right| \Delta r_e
        \right]
\end{equation}

\vspace{5mm}
Con las ecuaciones anteriores se llega a que \csvreader[no head]{data/calculos-3.csv}{}{
    $\varepsilon = \SI[exponent-mode = scientific]{\csvcolix \pm \csvcolx}{\farad\per\meter}$}, que
aún teniendo una incertidumbre relativa muy grande no es consistente con el
valor aceptado de $\varepsilon = \SI{8.85e-12}{\farad\per\meter}$.

