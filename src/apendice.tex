\section{Apéndice}

\subsection{Cálculo de incertidumbre de las mediciones}

Para los cálculos de incertidumbre en las mediciones realizadas con el 
multímetro se utilizó el manual de uso del DT-830B de Silver Electronics
\cite{multimetro}. Si bien este no es exactamente el mismo multímetro
utilizado en la práctica, es un instrumento común que podría encontrarse 
perfectamente en un laboratorio de la facultad, y asumimos que su tabla de
especificaciones es suficientemente similar a la del multímetro que utilizamos.

Por ejemplo, si se hubiera medido una tensión $V = \SI{9.00}{\volt}$, la
incertidumbre en la medición sería
$\Delta V = 0.5\%\, V + 2\times \SI{10}{\milli\volt} = \SI{0.065}{\volt}$ por
haber utilizado la escala de \SI{20}{\volt} para medir, con lo cual la medición
reportada sería de $V = \SI{9.00 \pm 0.07}{\volt}$.
