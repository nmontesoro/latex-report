\section{Segunda parte: amplificador no inversor}

Se arma el circuito de la fig. \ref{fig:2:esquema} en un \textit{protoboard}
con $R_1 = \SI{10}{\kilo\ohm}$ y $R_2 = \SI{15}{\kilo\ohm}$, un TL072 como
amplificador operacional y una fuente partida de $\SI{14}{\volt}$
alimentándolo. Luego se utilizan distintos valores de resistencia $R_L$
(47, 10 y \SI{1}{\kilo\ohm}) y se miden las tensiones $v_i$ y $v_o$ para
calcular la ganancia del circuito.

Hecho esto, se modifica el circuito para convertirlo en el seguidor de tensión
de la fig. \ref{fig:2:esquema-seguidor} (ver sección
\ref{sec:intro:opamp-noinversor}) y se vuelven a utilizar los
mismos valores de $R_L$ para hallar los distintos puntos de trabajo.

\begin{figure}[H]
    \centering
    \begin{subfigure}[b]{0.45\textwidth}
        \centering
        \begin{circuitikz}
    \node[op amp] at (0, 0) (opamp) {};

    \draw(opamp.+) -- ++(-0.2, 0) node[label={left:$v_+ = v_i$}]{}
    to[battery1, a=$V_1$] ++(0, -2) coordinate(cg)
    node[ground]{};

    \draw(opamp.-) -- ++(-0.7, 0) node[label={above:$v_-$}]{}
    to[R, a=$R_1$] ++(-1.5, 0)
    to[short] ++(-0.2, 0) coordinate(ci)
    to[short] (cg -| ci) node[ground]{};

    \draw(opamp.-) -- ++(-0.2, 0)
    to[short, *-] ++(0, 2)
    to[R=$R_2$] ++(2.8, 0) coordinate(co)
    to[short] (opamp.out -| co) coordinate(co1) -- (opamp.out);

    \draw(co1) 
    to[short, *-o] ++(0.4, 0) node[right]{$v_o$};

    \draw(co1)
    to[R=$R_L$] (cg -| co1) node[ground]{};

    \draw(opamp.up) -- ++(0, 0.2) node[vcc]{$+V_{cc}$};
    \draw(opamp.down) -- ++(0, -0.2) node[vee]{$-V_{cc}$};
\end{circuitikz}

        \caption{Amplificador no inversor}
        \label{fig:2:esquema}
    \end{subfigure}
    \hfill
    \begin{subfigure}[b]{0.45\textwidth}
        \centering
        \input{src/2/esquematico-seguidor.tikz}
        \caption{Amplificador seguidor de tensión}
        \label{fig:2:esquema-seguidor}
    \end{subfigure}
    \caption{Circuitos a implementar}
\end{figure}

