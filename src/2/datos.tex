\subsection{Datos obtenidos}

Para el amplificador no inversor, la tensión del nodo de entrada fue
$v_i = \SI{9.61 \pm 0.01}{\volt}$. Para los valores de resistencia 
$R_L$ de 47, 10 y \SI{1}{\kilo\ohm} se obtuvo siempre el mismo valor de
tensión a la salida del operacional: $v_o = \SI{20.90 \pm 0.01}{\volt}$.

Haciendo uso nuevamente de las ecuaciones \ref{ec:1-teoria:ganancia-experimental} y 
\ref{ec:1-teoria:err-ganancia-experimental} para el cálculo de la ganancia y del error, se
obtiene una ganancia $A = \num{2.174817898 \pm 0.003303660664}$.

En el caso del seguidor de tensión, $v_i = \SI{9.59 \pm 0.01}{\volt}$
y la salida registró $v_o = \SI{9.57 \pm 0.01}{\volt}$ para los tres valores
utilizados de $R_L$. La ganancia entonces es 
$A = \num{0.9979144943 \pm 0.002083331068}$.

