\subsection{Datos obtenidos}

Para el amplificador no inversor, la tensión del nodo de entrada fue
$v_i = \SI{9.61 \pm 0.01}{\volt}$. Para los valores de resistencia 
$R_L$ de 47, 10 y \SI{1}{\kilo\ohm} se obtuvo siempre el mismo valor de
tensión a la salida del operacional: $v_o = \SI{20.90 \pm 0.01}{\volt}$.

Haciendo uso nuevamente de las ecuaciones \ref{ec:1-teoria:ganancia-experimental} y 
\ref{ec:1-teoria:err-ganancia-experimental} para el cálculo de la ganancia y del error, se
obtiene una ganancia $A = \num{2.174817898 \pm 0.003303660664}$.

En el caso del seguidor de tensión, $v_i = \SI{9.59 \pm 0.01}{\volt}$
y la salida registró $v_o = \SI{9.57 \pm 0.01}{\volt}$ para los tres valores
utilizados de $R_L$. La ganancia entonces es 
$A = \num{0.9979144943 \pm 0.002083331068}$.

Con los datos anteriores y sabiendo que

\[
    I_{R_L} = \frac{v_o}{R_L} \quad\quad
    \Delta I_{R_L} = \frac{\Delta v_o}{R_L} + 
                     \left| \frac{-v_o}{{R_L}^2} \right| \Delta R_L 
\]
se calculan los valores de la tabla \ref{tab:2-datos:irl}.

\begin{table}[H]
    \centering
    \begin{tabular}{@{}r|rr@{}}
        \toprule
        $R_L$ (nominal, \si{\kilo\ohm}) & $R_L$ (medido, \si{\kilo\ohm}) & $I_{R_L}$
            (\si{\micro\ampere}) \\
        \midrule
        47 & \num{46.6 \pm 0.1} & \num{205.4 \pm 0.7} \\
        10 & \num{15.64 \pm 0.01} & \num{612 \pm 1} \\
         1 & \num{9.94 \pm 0.01} & \num{963 \pm 2} \\ \bottomrule
    \end{tabular}
    \caption{Corrientes por las resistencias $R_L$ en el seguidor de tensión
    ($v_i = \SI{9.57 \pm 0.01}{\volt}$)}
    \label{tab:2-datos:irl}
\end{table}
