\subsection{Datos obtenidos}

Los valores de tensión medidos para el amplificador no inversor se encuentran
en la tabla \ref{tab:2-datos:no-inversor}. Haciendo uso nuevamente de las 
ecuaciones \ref{ec:1-teoria:ganancia-experimental} y 
\ref{ec:1-teoria:err-ganancia-experimental} para el cálculo de la ganancia y 
del error, se hallan los valores de la tabla \ref{tab:2-datos:ganancia-no-inversor}.

\begin{table}[H]
    \centering
    \csvreader[
        tabular = r|ccc,
        table head = \toprule $R_L$ (nominal - \si{\kilo\ohm}) & $R_L$ (medido - \si{\kilo\ohm}) & $v_i$ (\si{\volt}) & $v_o$ (\si{\volt}) \\ \midrule,
        table foot = \bottomrule,
    ]{data/2/mediciones.csv}{}{
        \csvcoli & \num{\csvcolii \pm \csvcoliii} & \num{\csvcolvii \pm \csvcolviii} & \num{\csvcolv \pm \csvcolvi}
    }
    \caption{Valores medidos para el amplificador no inversor.}
    \label{tab:2-datos:no-inversor}
\end{table}

\begin{table}[H]
    \centering
    \csvreader[
        tabular = r|c,
        table head = \toprule $R_L$ (nominal - \si{\kilo\ohm}) & $A$ \\\midrule,
        table foot = \bottomrule,
    ]{data/2/mediciones.csv}{}{
        \num{\csvcoli} & \num{\csvcolix \pm \csvcolx}
    }
    \caption{Valores de ganancia medidos para el amplificador no inversor.}
    \label{tab:2-datos:ganancia-no-inversor}
\end{table}

Para el caso del seguidor de tensión se midieron los valores de la tabla 
\ref{tab:2-datos:seguidor}.

\begin{table}[H]
    \centering
    \csvreader[
        tabular = r|ccc,
        table head = \toprule $R_L$ (nominal - \si{\kilo\ohm}) & $v_i$ (\si{\volt}) & $v_o$ (\si{\volt}) & $A$ \\\midrule,
        table foot = \bottomrule,
    ]{data/2/mediciones-seguidor.csv}{}{
        \num{\csvcoli} & \num{\csvcolv \pm \csvcolvi} & \num{\csvcolvii \pm \csvcolviii} & \num{\csvcolix \pm \csvcolx}
    }
    \caption{Valores medidos para el seguidor de tensión.}
    \label{tab:2-datos:seguidor}
\end{table}

Con los datos anteriores y utilizando la ecuación
\ref{ec:2-datos:corriente-seguidor} se calculan los valores de la tabla
\ref{tab:2-datos:irl}.

\begin{equation}
    \label{ec:2-datos:corriente-seguidor}
    I_{R_L} = \frac{v_o}{R_L} \quad\quad
    \Delta I_{R_L} = \frac{\Delta v_o}{R_L} + 
                     \left| \frac{-v_o}{{R_L}^2} \right| \Delta R_L 
\end{equation}

\begin{table}[H]
    \centering
    \csvreader[
        tabular = r|cc,
        table head = \toprule $R_L$ (nominal - \si{\kilo\ohm}) & $R_L$ (medido - \si{\kilo\ohm}) & $I_{R_L}$ (\si{\milli\ampere}) \\\midrule,
        table foot = \bottomrule,
    ]{data/2/mediciones-seguidor.csv}{}{
        \num{\csvcoli} & \num{\csvcolii \pm \csvcoliii} & \num{\csvcolxi \pm \csvcolxii}
    }
    \caption{Corrientes por resistor $R_L$.}
    \label{tab:2-datos:irl}
\end{table}

