\subsection{Resolución teórica}

El circuito de la fig. \ref{fig:2:esquema} es prácticamente idéntico al de
la fig. \ref{fig:intro:opamp-no-inversor}, con el agregado del resistor $R_L$
que no modifica el valor de la tensión de salida.

Si se utiliza el modelo ideal planteado en la sección
\ref{sec:intro}, se puede despreciar el efecto del resistor
$R_1$ en el circuito de la fig. \ref{fig:2:esquema-seguidor}, ya que no
circulará corriente por esta rama debido a la alta resistencia de entrada del
operacional. De esta forma se obtiene el mismo circuito de la fig.
\ref{fig:intro:opamp-no-inversor}, con resistencia infinita entre la terminal
inversora y masa.

Para ambos casos, entonces, se puede utilizar la ec.
\ref{ec:intro:opamp-noinversor}. Al propagar el error se obtiene nuevamente
la ec. \ref{ec:1-teoria:err-ganancia}, motivo por el cual se obtiene una 
ganancia $A = \num{2.5 \pm 0.6}$ para el caso del amplificador no inversor,
y $A = \num{1.0 \pm 0.3}$ para el seguidor de
tensión\footnote{Considerando aquí $\Delta R_1 = 0$.}.
