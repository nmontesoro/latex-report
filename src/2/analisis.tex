\subsection{Análisis de datos}

Los gráficos de las figs. \ref{fig:2-analisis:ganancia-opamp} y 
\ref{fig:2-analisis:ganancia-seguidor} comparan los valores
de ganancia calculados teóricamente con los encontrados por simulación, y con
los valores calculados luego de tomar las mediciones. Se observa que los
valores experimentales y de simulación caen dentro del margen de error
esperado (descrito en la sección de cálculos teóricos).

\begin{figure}[H]
    \centering
    \input{img/2/ganancia.tikz}
    \caption{Ganancia del amplificador no inversor}
    \label{fig:2-analisis:ganancia-opamp}
\end{figure}

\begin{figure}[H]
    \centering
    \input{img/2/ganancia-seguidor.tikz}
    \caption{Ganancia del seguidor de tensión}
    \label{fig:2-analisis:ganancia-seguidor}
\end{figure}

Por otro lado, el gráfico de la fig. \ref{fig:2-analisis:recta-carga} muestra
cómo quedaría la recta de carga del seguidor de tensión teniendo en cuenta
los valores de tensión y resistencia medidos experimentalmente. Como puede
verse, se trata de una recta cuya pendiente tiende a $\infty$, tal cual se
obtuvo en la simulación con LTSpice. De haber usado valores de $R_L$ mucho
menores, en torno a los $\SI{200}{\ohm}$, hubiera sido posible comprobar si
el modelo de LTSpice es válido para tensiones de salida
$v_o < \SI{9.5}{\volt}$ (si es que el seguidor de tensión sigue proveyendo
un valor cercano a $v_i$ en la salida, o si efectivamente ocurre lo descrito
en la fig. \ref{fig:2:recta-spice}).

\begin{figure}[H]
    \centering
    \input{img/2/recta-carga.tikz}
    \caption{Recta de carga del seguidor de tensión}
    \label{fig:2-analisis:recta-carga}
\end{figure}

Debido al análisis realizado en los párrafos anteriores, es posible concluír
que el modelo ideal del amplificador operacional presentado en la sección
\ref{sec:intro} provee una aproximación muy cercana a la realidad, 
especialmente cuando se trata de calcular ganancias.
