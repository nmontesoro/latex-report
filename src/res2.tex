\question{
    Grafique el caudal en función del diámetro del orificio, en escala 
logarítmica.
}

\answer{
    El gráfico entregado por el software es uno de $F$ vs. $t$. Dado que 
    $F = ma$ y teniendo en cuenta que la fuerza medida es el peso, se puede
    decir que $F = P = mg$ y que, al menos para los datos que manejamos, esta
    función tiene forma de recta. Por lo tanto, si $A$ y $B$ son la pendiente y
    ordenada de esta recta, se puede decir que

\begin{equation}
    \label{ec:2:peso}
    F = mg \approx At + B
\end{equation}

    Despejando $m$ de la ecuación \ref{ec:2:peso} y derivando respecto del 
    tiempo se obtiene la ecuación del caudal másico:

\begin{equation}
    \label{ec:2:caudal-masico}
    \odv{m}{t} = \frac{A}{g}
\end{equation}

    Entonces para encontrar el valor del caudal másico bastaría con dividir
    la pendiente $A$ entregada por el software por el valor de la constante de
    gravedad $g$. Para cada diámetro de orificio se tomaron los valores de la 
    tabla \ref{tab:datos:mediciones-pendientes} y se calculó un único valor de 
    pendiente como $m = \bar{m} \pm 1.5\, \sigma_m$. Luego se dividió cada 
    pendiente por el valor de la constante de la gravedad\footnote{Se utilizó $g = \SI{9.80665}{\meter\per\second\squared}$ y se 
despreció la incertidumbre en este valor.}. Los resultados de
    estas operaciones se encuentran resumidos en la tabla \ref{tab:2:caudales},
    y fueron utilizados junto con los diámetros de la tabla 
    \ref{tab:datos:diametros} para construir el gráfico de la figura 
    \ref{fig:2:caudales}.

\begin{table}[H]
    \centering
    \csvreader[
        tabular = r|c,
        table head = \toprule Tapa & Caudal (\si{\kg\meter\per\second\cubed}) \\\midrule,
        table foot = \bottomrule,
    ]{data/caudales.csv}{}{
        \csvcoli & \num{\csvcolii \pm \csvcoliii}
    }
    \caption{Caudales másicos obtenidos del software.}
    \label{tab:2:caudales}
\end{table}

\begin{figure}[H]
    \centering
    \input{img/caudales.tikz}
    \caption{Caudales másicos vs. diámetro de la tapa.}
    \label{fig:2:caudales}
\end{figure}

}
